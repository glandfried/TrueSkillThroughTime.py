\documentclass[a4paper,10pt]{article}
\usepackage[utf8]{inputenc}
\input{tex/encabezado.tex}

\input{tex/tikzlibrarybayesnet.code.tex} % Para que ande se necesita copiar el archivo  tikzlibrarybayesnet.code.tex en la misma carpeta

\usetikzlibrary{arrows}
\usetikzlibrary{shapes}
\usetikzlibrary{fit}
\usetikzlibrary{chains}

%\usetikzlibrary{shapes.geometric,positioning}


\title{ True Skill \\ Technical Report}

\author{Gustavo Landfried}
\affil{\small Universidad de Buenos Aires. Facultad de Ciencias Exactas y Naturales. Departamento de Computaci\'on. Buenos Aires, Argentina}
\affil[]{Correspondencia: \url{gustavolandfried@gmail.com}}

\begin{document}

\maketitle

\todo[inline]{pasar $\vartheta_1$ a $\vartheta^2_1$}

\section{Bayes}


El Teorema de Bayes provee un puente entre par\'ametros no observables y los datos observados. Permite reducir la incertidumbre de los par\'ametros no observados mediante la certidumbre de los datos observados. Sea $H$ la hip\'otesis y $D$ los datos, 

\begin{equation}
 P(H|D)=\frac {P(D|H)P(H)}{P(D)}
\end{equation}

\begin{itemize}
\item $P(H)$, el \emph{prior}, representa la distribuci\'on de probabilidades de nuestras hip\'otesis. Permite capturar nuestros supuestos iniciales, el estado del conocimiento sobre los par\'ametros no observables. 
\item $P(D|H)$, el \emph{likelihood}, es la distribuci\'on de probabilidades de los datos dada una hip\'otesis espec\'ifica.
\item $P(D)$, el \emph{prior predictive}, es la distribuci\'on sobre los datos en funci\'on del conocimiento previo y nuestro modelo
\begin{equation}
P(D)= \int P(D,H) \mathrm{d}H = \int P(D|H)P(H) \mathrm{d}H 
\end{equation}
\item $P(H|D)$, la \emph{posterior}, la actualizaci\'on de las hip\'otesis o par\'ametros incorporando la evidencia de de los datos.
\item $P(\overline{D}|D)$, el \emph{posterior predictive}, es una distribuci\'on de probabilidades sobre los datos, con el conocimiento actualizado por los datos anteriores. 
\begin{equation}
P(\overline{D}|D)= \int P(\overline{D}|H)P(H|D) \mathrm{d}H 
\end{equation}

\end{itemize}

\section{Computo anal\'itico de la distribuci\'on posterior} \label{sec:computoAnilitico}


Para calcular la posterior utilizamos un algoritmo gen\'erico de pasaje de mensajes para \emph{factor graphs} llamado \emph{sum-product algorithm}~\cite{kschischang2001-factorGraphsAndTheSumProductAlgorithm}.
Permite resolver funciones marginales eficientemente mediante el uso de la forma en la cual la funci\'on global se factoriza en el producto de funciones locales simples, cada una dependiente de una subconjunto de variables.

\subsection{Sum-product algorithm}

Un \emph{factor graph} es una grafo bipartito, (relaciones entre nodos variable $v$ y nodos factor $f$).
Los ejes del factor graph representan la relaci\'on matem\'atica ``nodo $v$ es argumento de nodo $f$''.
La estructura del grafo codifica la factorizaci\'on de la funci\'on global.
Pero adem\'as, cuando un grafo de factorizaci\'on (\emph{factor graph}) no contiene ciclos, el grafo codifica tambi\'en las expresiones aritm\'eticas mediante las cuales se puede computar las marginales asociadas a la funci\'on global.

En el caso de \texttt{TrueSkill}, la funci\'on representada por el factor graph es la distribuci\'on conjunta $p(\bm{s},\bm{p},\bm{t}|\bm{o},A)$ (Fig.\ref{factorGraph_trueskill}). 

\begin{figure}[H]
  \centering
  \scalebox{.9}{\input{modelos/trueskill_factorGraph_versionGeneral.tex}}
  \caption{\small Grafo bipartito de la factorizaci\'on (\emph{factor graph}) del modelo \texttt{Trueskill}}
  \label{factorGraph_trueskill}
\end{figure}

\paragraph{The Sum-Product Update Rule} El mensaje enviado desde un v\'ertice $v$ a trav\'es de un eje $e$ es el producto de la funci\'on local de $v$ (funci\'on indicadora si $v$ es un nodo variable) con todos los mensajes recibidos en $v$ a trav\'es de ejes \emph{distintos} de $e$, integrados para la variable asociada con $e$.

\vspace{0.3cm}

Sea $m_{x \rightarrow f}(x)$ el mensaje enviado por el nodo variable $x$ al nodo factor $f$, y $m_{f \rightarrow x}(x)$ el mensaje enviado por un nodo factor $f$ a un nodo variable $x$.
Sea $n(v)$ el conjunto de nodos vecinos al nodo $v$.
Luego, los mensajes pueden ser expresados del siguiente modo.
\begin{equation}\label{eq:m_v_f} 
m_{x \rightarrow f}(x) = \prod_{h \in n(x) \setminus \{f\} } m_{h \rightarrow x}(x)
\end{equation}
\begin{equation}\label{eq:m_f_v}  
m_{f \rightarrow x}(x) = \int \cdots \int \Big( f(\bm{x}) \prod_{h \in n(f) \setminus \{x\} } m_{h \rightarrow f}(h) \Big) \,  d\bm{x}_{\setminus x}
\end{equation}

donde $\bm{x} = \text{arg}(f)$ es el conjunto de argumentos de la funci\'on $f$. 
Luego, para calcular una marginal cualquiera,
\begin{equation}\label{eq:marginal}
g_i(x_i) = \prod_{h \in n(x_i)} m_{h \rightarrow x_i}
\end{equation}

\subsection{Propiedades}\label{sec:propiedades}

Las siguientes tres propiedades, junto con las reglas del \emph{sum-product algorithm}, es lo \'unico que se necesita para calcular la posterior anal\'itica del modelo bayesiano.

\subsubsection*{Multiplicaci\'on de normales}  
\begin{equation}\label{eq:multiplicacion_normales}
\begin{split}
 \int_{-\infty}^{\infty} N(x|\mu_x,\sigma_x^2)N(x|\mu_y,\sigma_y^2) \, dx  &  \overset{*}{=} \int_{-\infty}^{\infty}  \underbrace{N(\mu_x|\mu_y,\sigma_x^2+\sigma_y^2)}_{\text{constante}} N(x|\mu_{*},\sigma_{*}^2) dx \\
 & = N(\mu_x|\mu_y,\sigma_x^2+\sigma_y^2) \underbrace{\int_{-\infty}^{\infty}  N(x|\mu_{*},\sigma_{*}^2) dx}_{1} \\
 & = N(\mu_x|\mu_y,\sigma_x^2+\sigma_y^2) 
\end{split}
\end{equation}

Donde la igualdad destacada ($\overset{*}{=}$) se demuestra en la secci\'on~\ref{multiplicacion_normales} anexa.

\subsubsection*{Integrales con funci\'on indicadora}
\begin{equation}\label{eq:integral_con_indicadora} 
\begin{split}
 \int_{-\infty}^{\infty}  \int_{-\infty}^{\infty}  \mathbb{I}(x=h(y,z)) f(x) g(y)\, dx\, dy &=  \int_{-\infty}^{\infty} \int_{h(y,z)}^{h(y,z)} f(h(y,z)) g(y)\, dx\, dy\\
 & = \int_{-\infty}^{\infty} f(h(y,z)) g(y) dy 
 %& \propto \int f(h(y,z)) g(y) dy 
\end{split}
\end{equation}

\subsubsection*{Simetr\'ia de normales}
\begin{equation}\label{eq:simetria}
 N(x|\mu,\sigma^2) = N(\mu|x,\sigma^2) = N(-\mu|-x,\sigma^2) = N(-x|-\mu,\sigma^2) 
\end{equation}

\subsubsection*{Derivada de la acumulada normal}
\begin{equation}\label{eq:phi_norm}
 \frac{\partial}{\partial x} \Phi(x|\mu,\sigma^2) = N(x|\mu,\sigma^2)
\end{equation}

\subsubsection*{Distribuci\'on normal estandarizada}
\begin{equation}\label{eq:estandarizar}
 X \sim N(\mu,\sigma^2) \Rightarrow \frac{X-\mu}{\sigma} \sim N(0,1)
\end{equation}



\subsection{Ejemplo 2 vs 2}

\begin{figure}[H]
  \centering
  \scalebox{.75}{\tikz{ %
        
      
        \node[factor] (fr) {} ;
        \node[const, right=of fr] (nfr) {$f_{r_1}$}; %
	
	\node[latent, above=of fr] (d) {$d_1$} ; %
        \node[factor, above=of d] (fd) {} ;
        \node[const, right=of fd] (nfd) {$f_{d_1}$}; %
	
        
        \node[latent, above=of fd,xshift=-0.75cm] (ta) {$t_a$} ; %
        \node[factor, left=of ta] (fta) {} ;
        \node[const, above=of fta] (nfta) {$f_{t_a}$}; %
        
        
        
        \node[latent, left=of fta,yshift=1cm] (p1) {$p_1$} ; %
        \node[factor, left=of p1] (fp1) {} ;
        \node[const, above=of fp1] (nfp1) {$f_{p_1}$}; %
        
        \node[latent, left=of fp1] (s1) {$s_1$} ; %
        \node[factor, left=of s1] (fs1) {} ;
	\node[const, above=of fs1] (nfs1) {$f_{s_1}$}; %
     
        \node[latent, left=of fta,yshift=-1cm] (p2) {$p_2$} ; %
        \node[factor, left=of p2] (fp2) {} ;
        \node[const, above=of fp2] (nfp2) {$f_{p_2}$}; %
        
        \node[latent, left=of fp2] (s2) {$s_2$} ; %
        \node[factor, left=of s2] (fs2) {} ;
	\node[const, above=of fs2] (nfs2) {$f_{s_2}$}; %
        
            
        \node[latent, above=of fd,xshift=0.75cm] (tb) {$t_b$} ; %
        \node[factor, right=of tb] (ftb) {} ;
        \node[const, above=of ftb] (nftb) {$f_{t_b}$}; %
        
        \node[latent, right=of ftb,yshift=1cm] (p3) {$p_3$} ; %
        \node[factor, right=of p3] (fp3) {} ;
        \node[const, above=of fp3] (nfp3) {$f_{p_3}$}; %
        
        \node[latent, right=of fp3] (s3) {$s_3$} ; %
        \node[factor, right=of s3] (fs3) {} ;
	\node[const, above=of fs3] (nfs3) {$f_{s_3}$}; %
     
        \node[latent, right=of ftb,yshift=-1cm] (p4) {$p_4$} ; %
        \node[factor, right=of p4] (fp4) {} ;
        \node[const, above=of fp4] (nfp4) {$f_{p_4}$}; %
        
        \node[latent, right=of fp4] (s4) {$s_4$} ; %
        \node[factor, right=of s4] (fs4) {} ;
	\node[const, above=of fs4] (nfs4) {$f_{s_4}$}; %
     
        \edge[-] {fr} {d};
	\edge[-] {d} {fd};
	
        \edge[-] {fd} {ta};
        \edge[-] {ta} {fta};
        \edge[-] {fta} {p1};
        \edge[-] {p1} {fp1};
        \edge[-] {fp1} {s1};
        \edge[-] {s1} {fs1};
        \edge[-] {fta} {p2};
        \edge[-] {p2} {fp2};
        \edge[-] {fp2} {s2};
        \edge[-] {s2} {fs2};
        	
	\edge[-] {fd} {tb};
        \edge[-] {tb} {ftb};
        \edge[-] {ftb} {p3};
        \edge[-] {p3} {fp3};
        \edge[-] {fp3} {s3};
        \edge[-] {s3} {fs3};
        \edge[-] {ftb} {p4};
        \edge[-] {p4} {fp4};
        \edge[-] {fp4} {s4};
        \edge[-] {s4} {fs4};
        
	
	\node[const, below=of nfr,xshift=7cm,yshift=-1cm] (dfr) { $fr_1 = \mathbb{I}(d_1>0)$}; %
	\node[const, left=of dfr,xshift=-0.5cm] (dfd) {$fd_1 = \mathbb{I}(d_1=t_a - t_b)$}; %
	\node[const, left=of dfd,xshift=-0.5cm] (dft) {$ft_e = \mathbb{I}(t_e = \sum_{i \in A_e} p_i)$}; %
        \node[const, left=of dft,xshift=-0.5cm] (dfp) {$fp_i = N(p_i;s_i,\beta^2)$}; %
        \node[const, left=of dfp,xshift=-0.5cm] (dfs) {$fs_i = N(s_i;\mu_i,\sigma^2)$}; %
 
	
	%\node[const, right= of r, xshift=1.2cm ,yshift=-2.1cm] (result-dist) {$r_{ab} \sim B\left(\Phi\left(\frac{\mu_a - \mu_b}{\sqrt{\beta_a^2+\beta_b^2}}\right)\right)$} ; %
	      
        }}
  \caption{\small Grafo bipartito de factorizaci\'on del modelo \texttt{TrueSkill} (Caso 2 vs 2)}
  \label{modelo_trueskill_2vs2}
\end{figure}


\subsubsection{Mensajes descendentes}

\paragraph{$\bm{m_{f_s \rightarrow s}(s)}:$}

\begin{equation}\label{eq:m_fs_s}
\begin{split}
 m_{f_{s_i} \rightarrow s_i}(s_i) & \overset{\hfrac{\text{eq}}{\ref{eq:m_f_v}}}{=} \int \dots \int f_{s_i}(\bm{x}) \prod_{h \in n(f_{s_i}) \setminus \{s_i\} } m_{h \rightarrow f_{s_i}}(h) d\bm{x}_{\setminus \{s_i\}}  \\
& \overset{\hfrac{\text{fig}}{\ref{modelo_trueskill_2vs2}}}{=} \int \dots \int N(s_i| \mu_i, \sigma_i^2) d\bm{x}_{\setminus \{s_i\} } \\
& \underset{*}{\overset{\hfrac{\text{eq}}{\ref{eq:m_f_v}}}{=}} N(s_i| \mu_i, \sigma_i^2)
\end{split}
\end{equation}

Notar que la igualdad señalada $\overset{*}{=}$ vale por definici\'on de la notaci\'on resumen (Eq.~\ref{eq:m_f_v}),

\begin{equation*}
\begin{split}
\bm{x}_{\setminus \{s_i\} } & = \text{arg}(f) \setminus \{s_i\} \\
&= \text{arg}(N(s_i| \mu_i, \sigma_i^2)) \setminus \{s_i\} \\
&= \{s_i\} \setminus \{s_i\} \\
& = \emptyset
\end{split}
\end{equation*}



\paragraph{$\bm{m_{s \rightarrow f_p}(s)}:$}

\begin{equation}\label{eq:m_s_fp}
 m_{s_i \rightarrow f_{p_i}}(s_i) \overset{\hfrac{\text{eq}}{\ref{eq:m_v_f}}}{=} \prod_{g \in n(s_i) \setminus  \{f_{p_i} \}} m_{g \rightarrow s_i} (s_i) \overset{\hfrac{\text{fig}}{\ref{modelo_trueskill_2vs2}}}{=} m_{f_{s_i} \rightarrow s_i}(s_i) \overset{\hfrac{\text{eq}}{\ref{eq:m_fs_s}}}{=}   N(s_i| \mu_i, \sigma_i^2)
\end{equation}



\paragraph{$\bm{m_{f_p \rightarrow p}(p)}:$}

\begin{equation}\label{eq:m_fp_p}
\begin{split}
 m_{f_{p_i} \rightarrow p_i}(p_i) & \overset{\hfrac{\text{eq}}{\ref{eq:m_f_v}}}{=} \int \dots \int f_{p_i}(\bm{x}) \prod_{h \in n(f_{p_i}) \setminus \{p_i\} } m_{h \rightarrow f_{p_i}}(h) d\bm{x}_{\setminus \{p_i\} }  \\
 & \overset{\hfrac{\text{fig}}{\ref{modelo_trueskill_2vs2}}}{\underset{\hfrac{\text{eq}}{\ref{eq:m_s_fp}}}{=}} \int \dots \int N(p_i| s_i, \beta^2) N(s_i| \mu_i, \sigma_i^2) d\bm{x}_{\setminus \{p_i\} } \\[0.3cm]
 & \overset{\hfrac{\text{eq}}{\ref{eq:m_f_v}}}{=} \int N(p_i| s_i, \beta^2) N(s_i| \mu_i, \sigma_i^2) ds_i \\
 & \overset{\hfrac{\text{eq}}{\ref{eq:simetria}}}{=} \int N(s_i|p_i, \beta^2) N(s_i| \mu_i, \sigma_i^2) ds_i \\ 
& \overset{\hfrac{\text{eq}}{\ref{eq:multiplicacion_normales}}}{=} N(p_i|\mu_i,\beta^2 + \sigma_i^2)
\end{split}
\end{equation}

\paragraph{$\bm{m_{p \rightarrow f_t}(p)}:$}

\begin{equation}\label{eq:m_p_ft}
\begin{split}
 m_{p_i \rightarrow f_{t_e}}(p_i) & \overset{\hfrac{\text{eq}}{\ref{eq:m_v_f}}}{=} \prod_{g \in n(p_i) \setminus  \{f_{t_e} \}} m_{g \rightarrow p_i} (p_i) \\ 
 & \overset{\hfrac{\text{fig}}{\ref{modelo_trueskill_2vs2}}}{=} m_{f_{p_i} \rightarrow p_i}(p_i) \overset{\hfrac{\text{eq}}{\ref{eq:m_fp_p}}}{=} N(p_i|\mu_i,\beta^2 + \sigma_i^2)
\end{split}
\end{equation}

\paragraph{$\bm{m_{f_t \rightarrow t}(t)}:$}

\begin{equation}
\begin{split}
 m_{f_{t_e} \rightarrow t_e}(t_e) & \overset{\hfrac{\text{eq}}{\ref{eq:m_f_v}}}{=} \int \dots \int f_{t_e}(\bm{x}) \prod_{h \in n(f_{t_e}) \setminus \{t_e\} } m_{h \rightarrow f_{t_e}}(h) d\bm{x}_{\setminus \{t_e\} }  \\
 & \overset{\hfrac{\text{fig}}{\ref{modelo_trueskill_2vs2}}}{\underset{\hfrac{\text{eq}}{\ref{eq:m_p_ft}}}{=}} \int \dots \int \mathbb{I}(t_e = p_i + p_j) N(p_i|\mu_i,\beta^2 + \sigma_i^2)N(p_j|\mu_j,\beta^2 + \sigma_j^2) d\bm{x}_{\setminus \{t_e\} }\\[0.3cm]
 & \overset{\hfrac{\text{eq}}{\ref{eq:m_f_v}}}{=} \iint \mathbb{I}(t_e = p_i + p_j) N(p_i|\mu_i,\beta^2 + \sigma_i^2)N(p_j|\mu_j,\beta^2 + \sigma_j^2) dp_idp_j \\
 & \overset{\hfrac{\text{eq}}{\ref{eq:integral_con_indicadora}}}{=} \int N(p_i|\mu_i,\beta^2 + \sigma_i^2) N(t_e - p_i|\mu_j,\beta^2 + \sigma_j^2) dp_i   \\
 & \overset{\hfrac{\text{eq}}{\ref{eq:simetria}}}{=} \int N(p_i|\mu_i,\beta^2 + \sigma_i^2) N(p_i|t_e - \mu_j,\beta^2 + \sigma_j^2) dp_i \\
 & \overset{\hfrac{\text{eq}}{\ref{eq:multiplicacion_normales}}}{=} N(t_e|\mu_i+\mu_j,2\beta^2 + \sigma_i^2 + \sigma_j^2)
\end{split}
\end{equation}

\vspace{0.3cm}

General (por inducci\'on)
\begin{equation}\label{eq:m_ft_t}
 m_{f_{t_e} \rightarrow t_e}(t_e) =  N \Big( t_e | \underbrace{\sum_{i\in A_e } \mu_i}_{\hfrac{\text{Habilidad}}{\text{de equipo}} \ \mu_e}, \underbrace{\sum_{i \in A_e} \beta^2 + \sigma_i^2}_{\hfrac{\text{Varianza}}{\text{de equipo}} \ \sigma_e^2} \Big) = N(t_e | \mu_e, \sigma_e^2)
\end{equation}

Ver anexo~\ref{suma_normales_induccion}, la secci\'on sobre suma de $n$ normales.

\paragraph{$\bm{m_{t \rightarrow f_d}(t)}:$}

\begin{equation}\label{eq:m_t_fd}
\begin{split}
m_{t_e \rightarrow f_{d_{k}}}(d_{k}) & \overset{\hfrac{\text{eq}}{\ref{eq:m_v_f}}}{=} \prod_{g \in n(t_e) \setminus  \{f_{d_{k}} \}} m_{g \rightarrow t_e} (t_e) \\[0.3cm] 
 & \overset{\hfrac{\text{fig}}{\ref{modelo_trueskill_2vs2}}}{=} m_{f_{t_e} \rightarrow t_e}(t_e) \overset{\hfrac{\text{eq}}{\ref{eq:m_ft_t}}}{=} N(t_e| \sum_{i \in A_e} \mu_i, \sum_{i \in A_e} \beta^2 + \sigma_i^2) \overset{\hfrac{\text{eq}}{\ref{eq:m_ft_t}}}{=} N(t_e|\mu_e,\sigma_e^2)
\end{split}
\end{equation}

\paragraph{$\bm{m_{f_d \rightarrow d}(d)}:$}

\begin{equation}
 \begin{split}
  m_{f_{d_1} \rightarrow d_1}(d_1) & \overset{\hfrac{\text{eq}}{\ref{eq:m_f_v}}}{=} \int \dots \int f_{d_1}(\bm{x}) \prod_{h \in n(f_{d_1}) \setminus \{d_1\} } m_{h \rightarrow f_{d_1}}(h) \, d\bm{x}_{\setminus \{d_1\} }  \\
  & \overset{\hfrac{\text{fig}}{\ref{modelo_trueskill_2vs2}}}{\underset{\hfrac{\text{eq}}{\ref{eq:m_t_fd}}}{=}} \int \int \mathbb{I}(d_1 = t_a - t_b) N(t_a| \mu_a, \sigma_a^2)  N(t_b| \mu_b, \sigma_b^2)  dt_adt_b \\[0.25cm]
  & \overset{\hfrac{\text{eq}}{\ref{eq:integral_con_indicadora}}}{=} \int N(d_1 + t_b| \mu_a, \sigma_a^2)  N(t_b| \mu_a, \sigma_b^2)  dt_b \\
  & \overset{\hfrac{\text{eq}}{\ref{eq:simetria}}}{=} \int N(t_b| \mu_a - d_1 , \sigma_a^2)  N(t_b| \mu_b, \sigma_b^2)  dt_b \\
  & \overset{\hfrac{\text{eq}}{\ref{eq:multiplicacion_normales}}}{=} N( \mu_a - d_1 | \mu_b, \sigma_a^2 +\sigma_b^2  ) \\
  & \overset{\hfrac{\text{eq}}{\ref{eq:simetria}}}{=} N( d_1 | \mu_a - \mu_b, \sigma_a^2 +\sigma_b^2  )
 \end{split}
\end{equation}

General

\begin{equation} \label{eq:m_fd_d}
 m_{f_{d_1} \rightarrow d_1}(d_1) = N\Bigg(d_1 \ | \ \underbrace{\sum_{i \in A_a} \mu_i - \sum_{i \in A_b} \mu_i}_{\hfrac{\text{Diferencia}}{\text{esperada}} \, (\delta)} \ , \  \underbrace{\sum_{i \in A_a \cup A_b} \beta^2 + \sigma_i^2}_{\hfrac{\text{Varianza}}{\text{total}} \, (\vartheta) } \Bigg) = N(d_1 | \delta, \vartheta)
\end{equation}

\paragraph{$\bm{m_{d \rightarrow f_r}(r)}:$}

\begin{equation}\label{eq:m_d_fr}
\begin{split}
m_{d \rightarrow f_r}(d) & \overset{\hfrac{\text{eq}}{\ref{eq:m_v_f}}}{=} \prod_{g \in n(t_e) \setminus  \{f_{d_{k}} \}} m_{g \rightarrow t_e} (t_e) \\[0.3cm] 
 & \overset{\hfrac{\text{fig}}{\ref{modelo_trueskill_2vs2}}}{=} m_{f_d \rightarrow d}(d) \overset{\hfrac{\text{eq}}{\ref{eq:m_fd_d}}}{=} N(d_1| \delta, \vartheta)
\end{split}
\end{equation}

\paragraph{$\bm{m_{f_r \rightarrow r}(r)}:$} (Caso ganador)

\begin{equation}\label{eq:m_fr_r}
\begin{split}
 m_{f_{r_1} \rightarrow r_1}(r_1) & \overset{\hfrac{\text{eq}}{\ref{eq:m_f_v}}}{=} \int \dots \int f_{r_1}(\bm{x}) \prod_{h \in n(f_{r_1}) \setminus \{r_1\} } m_{h \rightarrow f_{r_1}}(h) \, d\bm{x}_{\setminus \{r_1\} }  \\[0.2cm]
 & \overset{\hfrac{\text{fig}}{\ref{modelo_trueskill_2vs2}}}{\underset{\hfrac{\text{eq}}{\ref{eq:m_d_fr}}}{=}} \int \mathbb{I}(d_1 > 0) N(d_1 | \delta, \vartheta)  dd_1 \\[0.2cm]
 &  \overset{\hfrac{\text{eq}}{\ref{eq:estandarizar}}}{=} \int \mathbb{I}(d_1 > 0) N( \frac{d_1 - \delta}{\vartheta}) dd_1 \\[0.2cm]
 & = 1 - \Phi(\frac{-\delta}{\vartheta}) \\
 & = \Phi(\frac{\delta}{\vartheta})
\end{split}
\end{equation}



























\subsubsection{Mensajes ascendentes}

\paragraph{$\bm{m_{f_r \rightarrow d}(d)}:$}

\begin{equation}\label{eq:m_fr_d}
\begin{split}
m_{f_r \rightarrow d_1}(d_1) & \overset{\hfrac{\text{fig}}{\ref{modelo_trueskill_2vs2}}}{\underset{\hfrac{\text{eq}}{\ref{eq:m_f_v}}}{=}} \mathbb{I}(d_1 > 0)
\end{split}
\end{equation}


\paragraph{$\bm{m_{d \rightarrow f_d}(d)}:$}
\begin{equation}\label{eq:m_d_fd}
\begin{split}
m_{d_1 \rightarrow f_{d_1}}(d_1) & \overset{\hfrac{\text{eq}}{\ref{eq:m_v_f}}}{=} \prod_{g \in n(d_1) \setminus  \{f_{d_1} \}} m_{g \rightarrow d_1} (d_1) \\[0.3cm] 
 & \overset{\hfrac{\text{fig}}{\ref{modelo_trueskill_2vs2}}}{=} m_{f_r \rightarrow d_1}(d_1) \overset{\hfrac{\text{eq}}{\ref{eq:m_fr_d}}}{=} \mathbb{I}(d_1 > 0)
\end{split}
\end{equation}

\paragraph{$\bm{m_{f_{d_1} \rightarrow t_a}(t_a)}:$} (Caso ganador)
\begin{equation}\label{eq:m_fd_ta}
\begin{split}
m_{f_{d_1} \rightarrow t_a}(t_a) & \overset{\hfrac{\text{eq}}{\ref{eq:m_f_v}}}{=} \int \dots \int f_{d_1}(\bm{x}) \prod_{h \in n(f_{d_1}) \setminus \{t_a\} } m_{h \rightarrow f_{d_1}}(h) \, d\bm{x}_{\setminus \{t_a\} }  \\
&\overset{\hfrac{\text{fig}}{\ref{modelo_trueskill_2vs2}}}{\underset{\hfrac{\text{eq}}{\ref{eq:m_t_fd}}}{=}}  \int \dots \int \mathbb{I}(d_1 = t_a - t_b) \mathbb{I}(d_1 > 0) N(t_b | \mu_b , \sigma_b^2 )  \, d\bm{x}_{\setminus \{t_a\} } \\[0.1cm]
& \overset{\hfrac{\text{eq}}{\ref{eq:m_f_v}}}{=} \iint \mathbb{I}(d_1 = t_a - t_b) \mathbb{I}(d_1 > 0) N(t_b | \mu_b , \sigma_b^2 ) \, dd_1\,dt_b \\
& \overset{\hfrac{\text{eq}}{\ref{eq:integral_con_indicadora}}}{=} \int \mathbb{I}( t_a > t_b)  N(t_b | \mu_b , \sigma_b^2 ) \,dt_b  \\
& \overset{\hfrac{\text{fig}}{\ref{fig:m_fd_t}}}{=} \Phi (t_a| \mu_b, \sigma_b^2)  \overset{\hfrac{\mu_b}{\sigma_b}}{=}  \Phi \Big(t_a| \sum_{i \in A_b} \mu_i , \sum_{i \in A_b} \beta^2 + \sigma_i^2 \Big)
\end{split}
\end{equation}

\paragraph{$\bm{m_{f_{d_1} \rightarrow t_b}(t_b)}:$} (Caso perdedor)
\begin{equation}\label{eq:m_fd_tb}
\begin{split}
m_{f_{d_1} \rightarrow t_b}(t_b) &\overset{\hfrac{\text{eq}}{\ref{eq:m_f_v}}}{=} \int \dots \int f_{d_1}(\bm{x}) \prod_{h \in n(f_{d_1}) \setminus \{t_b\} } m_{h \rightarrow f_{d_1}}(h) \, d\bm{x}_{\setminus \{t_a\} }  \\
&\overset{\hfrac{\text{fig}}{\ref{modelo_trueskill_2vs2}}}{\underset{\hfrac{\text{eq}}{\ref{eq:m_t_fd}}}{=}}  \int \dots \int \mathbb{I}(d_1 = t_a - t_b) \mathbb{I}(d_1 > 0) N(t_a | \mu_a , \sigma_a^2 )  \, d\bm{x}_{\setminus \{t_b\} } \\[0.1cm]
&\overset{\hfrac{\text{eq}}{\ref{eq:m_f_v}}}{=} \iint \mathbb{I}(d_1 = t_a - t_b) \mathbb{I}(d_1 > 0)  N(t_a | \mu_a , \sigma_a^2 )  \, dd_1\,dt_a \\
&\overset{\hfrac{\text{eq}}{\ref{eq:integral_con_indicadora}}}{=} \int \mathbb{I}( t_a > t_b)   N(t_a | \mu_a , \sigma_a^2 ) \,dt_a \\
&\overset{\hfrac{\text{fig}}{\ref{fig:m_fd_t}}}{=} 1 - \Phi (t_b| \mu_a , \sigma_a^2 ) \overset{\hfrac{\mu_a}{\sigma_a}}{=} 1 - \Phi \Big(t_b| \sum_{i \in A_a} \mu_i , \sum_{i \in A_a} \beta^2 + \sigma_i^2 \Big)
\end{split}
\end{equation}

\begin{figure}[H]
\centering
  \begin{subfigure}[t]{0.48\textwidth}
  \includegraphics[width=\textwidth]{figures/m_d_ta.pdf}
  \caption{$m_{f_{d_1} \rightarrow t_a}(t_a)$: (Caso ganador)}
  \label{fig:m_fd_ta}
  \end{subfigure}
  \begin{subfigure}[t]{0.48\textwidth}
  \includegraphics[width=\textwidth]{figures/m_d_tb.pdf}
  \caption{$m_{f_{d_1} \rightarrow t_b}(t_b)$: (Caso perdedor)}
  \label{fig:m_fd_tb}
  \end{subfigure}
  \caption{Notar que en el caso ganador, $t_a$ es un valor fijo que entra como par\'ametro en la funci\'on $m_{f_{d_1} \rightarrow t_a}(t_a)$. El caso perdedor es an\'alogo.}
  \label{fig:m_fd_t}
\end{figure}

\paragraph{$\bm{m_{t_a \rightarrow f_{t_a}}(t_a)}:$} (Caso ganador)

\begin{equation}\label{eq:m_ta_ft}
\begin{split}
 m_{t_a \rightarrow f_{t_a}}(t_a) \overset{\hfrac{\text{eq}}{\ref{eq:m_v_f}}}{=} \prod_{g \in n(t_a) \setminus  \{f_{t_a} \}} m_{g \rightarrow t_a} (t_a)  \overset{\hfrac{\text{eq}}{\ref{eq:m_fd_ta}}}{=} \Phi(t_a|\mu_b,\sigma_b^2) \overset{\hfrac{\mu_b}{\sigma_b}}{=} \Phi \Big(t_a| \sum_{i \in A_b} \mu_i , \sum_{i \in A_b} \beta^2 + \sigma_i^2 \Big) 
\end{split}
\end{equation}

\paragraph{$\bm{m_{t_b \rightarrow f_{t_b}}(t_b)}:$} (Caso perdedor)
\begin{equation}\label{eq:m_tb_ft}
\begin{split}
 m_{t_b \rightarrow f_{t_b}}(t_b) \overset{\hfrac{\text{eq}}{\ref{eq:m_v_f}}}{=} \prod_{g \in n(t_b) \setminus  \{f_{t_b} \}} m_{g \rightarrow t_b} (t_b)  \overset{\hfrac{\text{eq}}{\ref{eq:m_fd_tb}}}{=} 1- \Phi(t_b|\mu_a,\sigma_a^2) \overset{\hfrac{\mu_a}{\sigma_a}}{=} 1 - \Phi \Big(t_b| \sum_{i \in A_a} \mu_i , \sum_{i \in A_a} \beta^2 + \sigma_i^2 \Big) 
\end{split}
\end{equation}



\paragraph{$\bm{m_{f_{t_a} \rightarrow p_1}(p_1)}:$} (Caso ganador)
\begin{equation}\label{eq:m_fta_p_inicial}
\begin{split}
m_{f_{t_a} \rightarrow p_1}(p_1)  &\overset{\hfrac{\text{eq}}{\ref{eq:m_f_v}}}{=} \int \dots \int f_{t_a}(\bm{x}) \prod_{h \in n(f_{t_a}) \setminus \{p_1\} } m_{h \rightarrow f_{t_a}}(h) \, d\bm{x}_{\setminus \{p_1\} }  \\
&\overset{\hfrac{\text{fig}}{\ref{modelo_trueskill_2vs2}}}{\underset{\hfrac{\text{eq}}{\ref{eq:m_ta_ft}}}{=}} \int \dots \int \mathbb{I}( t_a = p_1 + p_2) N(p_2| \mu_2, \beta^2 + \sigma_2^2 ) \, \Phi (t_a| \mu_b , \sigma_b^2 ) \, d\bm{x}_{\setminus \{p_1\} }\\[0.1cm]
&\overset{\hfrac{\text{eq}}{\ref{eq:m_f_v}}}{=} \iint \mathbb{I}( t_a = p_1 + p_2) \, N(p_2| \mu_2, \beta^2 + \sigma_2^2 ) \, \Phi (t_a| \mu_b , \sigma_b^2 ) \, dt_a dp_2 \\
&\overset{\hfrac{\text{eq}}{\ref{eq:integral_con_indicadora}}}{=} \int  \, N(p_2| \mu_2, \beta^2 + \sigma_2^2 ) \, \Phi (p_1 + p_2| \mu_b , \sigma_b^2 ) \, dp_2 \\
&\overset{\hfrac{\text{eq}}{\ref{eq:simetria}}}{=} \int  N(p_2| \mu_2, \beta^2 + \sigma_2^2 ) \, \Phi (p_1 | \mu_b - p_2 , \sigma_b^2) \, dp_2 \\
&= \kappa(p_1)
\end{split}
\end{equation}

La derivada de la funci\'on de distribución acumulada $\Phi(\,)$ es el valor de la densidad de la función de probabilidad $N(\,)$. Con esta idea en mente, tomamos la derivada de ambos lados de la igualdad:

\begin{equation}\label{eq:ta-p_derivada}
\begin{split}
\frac{\partial\kappa(x)}{\partial x} &= \frac{\partial}{\partial x} \int  N(y| \mu_y, \sigma_y^2 ) \,   \Phi (x | \mu_x -y , \sigma_x^2 ) \, dy \\
&= \int  N(y| \mu_y, \sigma_y^2 ) \, \frac{\partial}{\partial x} \,\Phi (x| \mu_x - y, \sigma_x^2 )  \, dy   \\
& \overset{\hfrac{\text{eq}}{\ref{eq:phi_norm}}}{=} \int  N(y| \mu_y, \sigma_y^2 ) \, N(x| \mu_x -y , \sigma_x^2)  \, dy  \\
& \overset{\hfrac{\text{eq}}{\ref{eq:simetria}}}{=} \int  N(y| \mu_y, \sigma_y^2 ) \, N(y| \mu_x  -x , \sigma_x^2)  \, dy  \\
&\overset{\hfrac{\text{eq}}{\ref{eq:multiplicacion_normales}}}{\underset{\hfrac{\text{eq}}{\ref{eq:simetria}}}{=}} N(x| \mu_x - \mu_y, \sigma_x^2 + \sigma_y^2) 
\end{split}
\end{equation}

Luego

\begin{equation}\label{eq:m_fta_p}
 m_{f_{t_a} \rightarrow p_1}(p_1) \overset{\hfrac{\text{eq}}{\ref{eq:m_fta_p_inicial}}}{\underset{\hfrac{\text{eq}}{\ref{eq:ta-p_derivada}}}{=}}  \Phi(p_1| \mu_b - \mu_2, \beta^2 + \sigma_2^2 + \sigma_b^2)  \overset{\hfrac{\mu_b}{\sigma_b}}{=}  \Phi\Big(p_1| \sum_{i \in A_b} \mu_i - \mu_2, \beta^2 + \sigma_2^2 + \sum_{i \in A_b} \beta^2 + \sigma_i^2 \Big) 
\end{equation}

\paragraph{$\bm{m_{f_{t_b} \rightarrow p_3}(p_3)}:$} (Caso perdedor)
\begin{equation}\label{eq:m_ftb_p}
\begin{split}
m_{f_{t_b} \rightarrow p_3}(p_3)&\overset{\hfrac{\text{eq}}{\ref{eq:m_f_v}}}{=} \int \dots \int f_{t_a}(\bm{x}) \prod_{h \in n(f_{t_a}) \setminus \{p_1\} } m_{h \rightarrow f_{t_a}}(h) \, d\bm{x}_{\setminus \{p_1\} }  \\
&\overset{\hfrac{\text{fig}}{\ref{modelo_trueskill_2vs2}}}{\underset{\hfrac{\text{eq}}{\ref{eq:m_tb_ft}}}{=}} \int \dots \int \mathbb{I}( t_b = p_3 + p_4) \, (1-\Phi (t_b| \mu_a , \sigma_a^2 )) \, N(p_4| \mu_4, \beta^2 + \sigma_4^2 ) \, d\bm{x}_{\setminus \{p_3\} }\\[0.1cm]
&\overset{\hfrac{\text{eq}}{\ref{eq:m_f_v}}}{=} \iint \mathbb{I}( t_b = p_3 + p_4) N(p_4| \mu_4, \beta^2 + \sigma_4^2 )  (1 - \Phi (t_b| \mu_a , \sigma_a^2) )\, dt_b dp_4 \\
&\overset{\hfrac{\text{eq}}{\ref{eq:integral_con_indicadora}}}{\underset{\hfrac{\text{eq}}{\ref{eq:simetria}}}{=}} \int N(p_4| \mu_4, \beta^2 + \sigma_4^2 )  (1 - \Phi (p_3 | \mu_a - p_4 , \sigma_a^2 ) ) \,  dp_4 \\
& =  \underbrace{\int N(p_4| \mu_4, \beta^2 + \sigma_4^2 )dp_4}_{1}  -  \underbrace{\int N(p_4| \mu_4, \beta^2 + \sigma_4^2 ) \Phi (p_3 | \mu_a - p_4 , \sigma_a^2 ) ) \, dp_4}_{\kappa(p_3)} \\
&\overset{\hfrac{\text{eq}}{\ref{eq:ta-p_derivada}}}{=} 1 - \Phi(p_3, \mu_a  - \mu_4, \beta^2 + \sigma_4^2 + \sigma_a^2)\\[0.1cm]
&\overset{\hfrac{\mu_a}{\sigma_a}}{=} 1 - \Phi\Big(p_3, \sum_{i \in A_a} \mu_i  - \mu_4, \beta^2 + \sigma_4^2 + \sum_{i \in A_a} \beta^2 + \sigma_i^2  \Big)
\end{split}
\end{equation}

\paragraph{$\bm{m_{p_1 \rightarrow f_{p_1}}(s_1)}:$} (Caso ganador)

\begin{equation}\label{eq:m_p1_fp}
\begin{split}
 m_{p_1 \rightarrow f_{p_1}}(p_1) \overset{\hfrac{\text{eq}}{\ref{eq:m_v_f}}}{=} \prod_{g \in n(p_1) \setminus  \{f_{p_1} \}} m_{g \rightarrow p_1} (p_1)  \overset{\hfrac{\text{eq}}{\ref{eq:m_fta_p}}}{=}  \Phi(p_1| \mu_b - \mu_2, \beta^2 + \sigma_2^2 + \sigma_b^2) 
\end{split}
\end{equation}


\paragraph{$\bm{m_{p_3 \rightarrow f_{p_3}}(s_1)}:$} (Caso perdedor)

\begin{equation}\label{eq:m_p3_fp}
\begin{split}
 m_{p_3 \rightarrow f_{p_3}}(p_3) \overset{\hfrac{\text{eq}}{\ref{eq:m_v_f}}}{=} \prod_{g \in n(p_3) \setminus  \{f_{p_3} \}} m_{g \rightarrow p_3} (p_3)  \overset{\hfrac{\text{eq}}{\ref{eq:m_ftb_p}}}{=}  1 - \Phi(p_3, \mu_a  - \mu_4, \beta^2 + \sigma_4^2 + \sigma_a^2) 
\end{split}
\end{equation}

\paragraph{$\bm{m_{f_{p_1} \rightarrow s_1}(s_1)}:$} (Caso ganador)
\begin{equation}\label{eq:m_fp_s1}
\begin{split}
m_{f_{p_1} \rightarrow s_1}(s_1) & \overset{\hfrac{\text{eq}}{\ref{eq:m_f_v}}}{=} \int \dots \int f_{p_1}(\bm{x}) \prod_{h \in n(f_{p_1}) \setminus \{s_1\} } m_{h \rightarrow f_{p_1}}(h) \, d\bm{x}_{\setminus \{s_1\} }  \\
&\overset{\hfrac{\text{fig}}{\ref{modelo_trueskill_2vs2}}}{\underset{\hfrac{\text{eq}}{\ref{eq:m_p1_fp}}}{=}} \int \dots \int N(p_1| s_1, \beta^2) \, \Phi(p_1| \mu_b - \mu_2, \beta^2 + \sigma_2^2 + \sigma_b^2 ) \, d\bm{x}_{\setminus \{s_1\} }
\\[0.1cm]
& \overset{\hfrac{\text{eq}}{\ref{eq:m_f_v}}}{\underset{\hfrac{\text{eq}}{\ref{eq:simetria}}}{=}} \int N(p_1| s_1, \beta^2) \, \Phi(\mu_2| \mu_b -  p_1, \beta^2 + \sigma_2^2 + \sigma_b^2) \, dp_1 \\[0.1cm]
&\overset{\hfrac{\text{eq}}{\ref{eq:ta-p_derivada}}}{=} \Phi(s_1| \mu_b - \mu_2, 2\beta^2 + \sigma_2^2 + \sigma_b^2)  
\end{split}
\end{equation}

General (N vs N)
\begin{equation}\label{eq:m_fp_s1_gral}
\begin{split}
m_{f_{p_1} \rightarrow s_1}(s_1) & \overset{\hfrac{\text{eq}}{\ref{eq:m_fp_s1}}}{=} \Phi(s_1| \mu_b - \mu_a + \mu_1, \sigma_b^2 +\sigma_a^2 - \sigma_1^2 )  \\
& \overset{\hfrac{\mu_b}{\sigma_b}}{=} \Phi\Big(s_1| \underbrace{\sum_{i \in A_b} \mu_i - \sum_{i \in A_a} \mu_i }_{-\hfrac{\text{Diferencia}}{\text{esperada}} \, -\delta = \delta } + \mu_1 , \underbrace{\sum_{i \in A_b \cup A_a} \beta^2 + \sigma_i^2}_{\hfrac{\text{Varianza}}{\text{total}} \, \vartheta^2} - \sigma_1^2   \Big) \\
& \overset{\hfrac{\delta}{\vartheta}}{=} \Phi(s_1|-\delta + \mu_1,\vartheta^2-\sigma_1^2) \\
& \overset{\hfrac{\text{eq}}{\ref{eq:simetria}}}{=} 1- \Phi(0| \underbrace{\delta - \mu_1 + s_1}_{\hfrac{\text{Diferencia esperada}}{\text{parametrizada}} \, \delta_1(s_1)},\underbrace{\vartheta^2-\sigma_1^2}_{\vartheta_1^2}) \\
& \overset{\hfrac{\delta_1}{\vartheta_1}}{=} 1- \Phi(0|\delta_1(s_1),\vartheta_1^2) \\
&\overset{\hfrac{\text{eq}}{\ref{eq:estandarizar}}}{=} 1- \Phi\Big(\frac{0-\delta_1(s_1)}{\vartheta_1}\Big)\\
&\overset{\hfrac{\text{eq}}{\ref{eq:simetria}}}{=} \Phi\Big(\frac{\delta_1(s_1)}{\vartheta_1}\Big)
\end{split}
\end{equation}

\textbf{Nota}: el mensaje $m_{f_{p_1} \rightarrow s_1}(s_1)$ computa la ``probabilidad de ganar parametrizada'', esto es la probabilidad de ganar si conociéramos la habilidad del jugador. Si conocemos la habilidad del jugador entonces hay que eliminar la varianza de la distribución de creencias de la varianza total (lo que hacemos en $\vartheta_1$) y hay que remplazar la media de la distribución de creencias por la verdadera habilidad en la diferencia esperada (lo que hacemos en $\delta_1(s_1)$).

\paragraph{$\bm{m_{f_{p_3} \rightarrow s_3}(s_3)}:$}

\begin{equation}\label{eq:m_fp_s3}
\begin{split}
m_{f_{p_3} \rightarrow s_3}(s_3) & \overset{\hfrac{\text{eq}}{\ref{eq:m_f_v}}}{=} \int \dots \int f_{p_3}(\bm{x}) \prod_{h \in n(f_{p_3}) \setminus \{s_3\} } m_{h \rightarrow f_{p_3}}(h) \, d\bm{x}_{\setminus \{s_3\} }  \\
&\overset{\hfrac{\text{fig}}{\ref{modelo_trueskill_2vs2}}}{\underset{\hfrac{\text{eq}}{\ref{eq:m_p3_fp}}}{=}} \int \dots \int N(p_3| s_3, \beta^2) (1 - \Phi(p_3, \mu_a  - \mu_4, \beta^2 + \sigma_4^2 + \sigma_a^2)) \, d\bm{x}_{\setminus \{s_3\} }\\
& \overset{\hfrac{\text{eq}}{\ref{eq:m_f_v}}}{\underset{\hfrac{\text{eq}}{\ref{eq:simetria}}}{=}} \int N(p_3| s_3, \beta^2) (1 - \Phi(\mu_4, \mu_a  - p_3, \beta^2 + \sigma_4^2 + \sigma_a^2)) \, dp_3 \\
&=\int N(p_3| s_3, \beta^2) \, dp_3 -  \int N(p_3| s_3, \beta^2)  \Phi(\mu_4, \mu_a  - p_3, \beta^2 + \sigma_4^2 + \sigma_a^2) \, dp_3 \\
&\overset{\hfrac{\text{eq}}{\ref{eq:ta-p_derivada}}}{=} 1 - \Phi\Big(s_3| \mu_a-  \mu_4, 2\beta^2 + \sigma_4^2 + \sigma_a^2 \Big)
\end{split}
\end{equation}

General (N vs N)

\begin{equation}
\begin{split}
m_{f_{p_3} \rightarrow s_3}(s_3) & \overset{\hfrac{\text{eq}}{\ref{eq:m_fp_s3}}}{=} 1 - \Phi\Big(s_3| \underbrace{\mu_a-\mu_b}_{\delta}+\mu_3, \underbrace{\sigma_a^2 + \sigma_b^2}_{\vartheta^2} - \sigma_3^2 \Big) \\
& \overset{\hfrac{\delta}{\vartheta}}{=} 1 - \Phi\Big(s_3| \delta +\mu_3, \vartheta^2- \sigma_3^2 \Big) \overset{\hfrac{\text{eq}}{\ref{eq:simetria}}}{=} \Phi\Big(0| \underbrace{-\delta-\mu_3+s_3}_{\delta_3(s_3)}, \underbrace{\vartheta^2- \sigma_3^2}_{\vartheta_3^2} \Big) \\
& \overset{\hfrac{\delta_3}{\vartheta_3}}{=} \Phi(0|\delta_3(s_3),\vartheta_3^2)  \overset{\hfrac{\text{eq}}{\ref{eq:estandarizar}}}{=}  \Phi\left(\frac{0-\delta_3(s_3)}{\vartheta_3}\right) \\
& \overset{\hfrac{\text{eq}}{\ref{eq:simetria}}}{=} \Phi\Big(\frac{-\delta_3(s_3)}{\vartheta_3}\Big)
\end{split}
\end{equation}

\textbf{Nota}: el mensaje $m_{f_{p_3} \rightarrow s_3}(s_3)$ computa la ``probabilidad de perder parametrizada'' (que es la misma que la probabilidad de ganar de su contrincante). Si conociéramos la habilidad del jugador entonces hay que eliminar la varianza de la distribución de creencias de la varianza total (lo que hacemos en $\vartheta_3$) y hay que remplazar la media de la distribución de creencias por la verdadera habilidad en la diferencia esperada (lo que hacemos en $\delta_3(s_3)$).

\subsubsection{Posterior ganador y perdedor}
\paragraph{Ganador}
\begin{equation}\label{eq:posterior_ganador}
 p(s_1|o,A) \overset{\hfrac{\text{eq}}{\ref{eq:marginal}}}{=} \prod_{h \in n(x_i)} m_{h \rightarrow x_i} \overset{\hfrac{\text{fig}}{\ref{modelo_trueskill_2vs2}}}{\underset{\hfrac{\text{eq}}{\ref{eq:m_fp_s1_gral}}}{=}}  N(s_1| \mu_1, \sigma_1^2)  \Phi\left(\frac{\delta_1(s_1)}{\vartheta_1}\right)
\end{equation}


\begin{figure}[H]
\centering
  \begin{subfigure}[t]{0.48\textwidth}
  \includegraphics[page=1,width=\textwidth]{figures/posterior_ganador.pdf}
  \caption{}
  \label{posterior_ganador_image}
  \end{subfigure}
  \begin{subfigure}[t]{0.48\textwidth}
  \includegraphics[page=2,width=\textwidth]{figures/posterior_ganador.pdf}
  \caption{}
  \label{posterior_ganador_distribution}
  \end{subfigure}
  \caption{Posterior ganador}
  \label{posterior_ganador}
\end{figure}

\paragraph{Perdedor}

\begin{equation}\label{eq:posterior_perdedor}
 p(s_1|o,A) \overset{\hfrac{\text{eq}}{\ref{eq:marginal}}}{=} \prod_{h \in n(x_i)} m_{h \rightarrow x_i} \overset{\hfrac{\text{fig}}{\ref{modelo_trueskill_2vs2}}}{\underset{\hfrac{\text{eq}}{\ref{eq:m_fp_s1_gral}}}{=}}  N(s_1| \mu_1, \sigma_1^2)  \Phi\left(\frac{-\delta_1(s_1)}{\vartheta_1}\right)
\end{equation}


\begin{figure}[H]
\centering
  \begin{subfigure}[t]{0.48\textwidth}
  \includegraphics[page=1,width=\textwidth]{figures/posterior_perdedor.pdf}
  \caption{}
  \label{posterior_perdedor_image}
  \end{subfigure}
  \begin{subfigure}[t]{0.48\textwidth}
  \includegraphics[page=2,width=\textwidth]{figures/posterior_perdedor.pdf}
  \caption{}
  \label{posterior_perdedor_distribution}
  \end{subfigure}
  \caption{Posterior perdedor}
  \label{posterior_perdedor}
\end{figure}

\section{Prior predictive y evidencia}

La evidencia es otro nombre para el Prior predictive.

\begin{equation*}
\underbrace{P(\text{Belief}|\text{Data})}_{\text{\scriptsize Posterior}} = \frac{\overbrace{P(\text{Data}|\text{Belief})}^{\text{\scriptsize Likelihood}} \overbrace{P(\text{Belief})}^{\text{\scriptsize Prior}} }{\underbrace{P(\text{Data})}_{\hfrac{\hfrac{\text{\scriptsize Evidence or}}{\text{\scriptsize Average likelihood}}}{\text{\scriptsize Prior Predictive}} } }
\end{equation*}

 \begin{figure}[H]
\centering
  \begin{subfigure}[t]{0.48\textwidth}
  \includegraphics[page=1,width=\textwidth]{figures/evidence_win.pdf}
  \caption{Evidence}
  \label{fig:evidence_win}
  \end{subfigure}
  \begin{subfigure}[t]{0.48\textwidth}
  \caption{Prior predictive}
  \label{fig:prior_predictive_win}
  \end{subfigure}
  \caption{The Evidence and the Prior predictive are the same object, so they have the same area under the curve}
  \label{fig:evidence_prior_predictive}
\end{figure}
 
 Caso ganador
 
 \begin{equation}\label{eq:prior_predictive}
  P(r=\text{win}) = \prod_{h \in n(r)} m_{h \rightarrow r}  = m_{f_r \rightarrow r}(r) =  \Phi(\frac{\delta_1}{\vartheta_1}) 
 \end{equation}

 

\section{Draw}

\subsection{TrueSkill draw-solution}

If draws are permitted, the winning outcome $r_{(j)} < r_{(j+1)}$, requires $t_{r_{(j)}} > t_{r_{(j+1)}} + \epsilon$, and the draw outcome $r_{(j)}=r_{(j+1)}$ requires $|t_{r_{(j)}} > t_{r_{(j+1)}}| \leq \epsilon$ where $\epsilon$ is a draw margin.

\vspace{0.3cm}

The draw margin can be seted by the a given ``empirical draw probability''.
Para entender de donde obtener este valor, debemos recordar que la diferencia de rendimientos verdadera se generan a partir de una distribuci\'on normal centrada en las verdaderas habilidad y con la suma de los desvios asociacios a los rendimientos $\N( \sum_i s_i^{\text{true}} - \sum_j s_j^{\text{true}},(n_1+n_2)\beta^2)$.
Notar que esta distribuci\'on no depende de nuestras creencias, por eso no depende de nuestra incertidumbre $\sigma$. 

\begin{figure}[H]
\centering
  \includegraphics[width=0.5\textwidth]{figures/draw.pdf}
  \caption{La distirbuci\'on verdadera de diferencia entre equipos, $\N( \sum_i s_i^{\text{true}} - \sum_j s_j^{\text{true}},(n_1+n_2)\beta^2)$.}
  \label{fig:draw}
\end{figure}

Intuitivamente sabemos que la probabilidad de empate depende de la diferencia de habilidad.
Por eso, cuando hablamos de \textbf{una} probabilidad emp\'irica, en el fondo estamos pensando \textbf{la} probabilidad emp\'irica para una diferencia de habilidad determinada.
En particular \textbf{la probabilidad de empates para jugadores \emph{si tuviera misma habilidad}}.
En este punto nos estamos mordiendo la cola, porque estamos construyendo el modelo para estimar habilidades justamente porque no las conocemos.
En este punto es bueno remarcar que la probabilidad de empate para jugadores de misma habilidad, así como el $\beta$, es un valor en buen medida arbitraro de nuestro modelo.
Si lo quisieramos estimar, deberíamos proponer una distirbuci\'on de creencias a priori, y actualizarla con nuestro modelo.

\vspace{0.3cm}

Suponiendo que tenemos la probabilidad de empate para oponentes de misma habilidad, para mantener estable el \'area ante diferentes escenarios, el $\epsilon$ debe poder adaptarse.
El \'unico par\'ametro libre en este caso es la cantidad de jugadores en los equipos, que modifica el ancho de la distribuci\'on de la diferencia de habilidad.

\begin{figure}[H]
\centering
  \includegraphics[page=2,width=0.5\textwidth]{figures/draw.pdf}
  \caption{Distribuci\'on de probabilidades de diferencias de rendimientos, y área correspondiente a la probabilidad de empate $0.25$}
  \label{fig:draw}
\end{figure}

Por eso, el $\epsilon$ se dice que depende de, 

\begin{equation}
 \text{Draw probability} = \Phi(\frac{\epsilon}{\sqrt{n_1+n_2}\beta}) - \Phi(\frac{-\epsilon}{\sqrt{n_1+n_2}\beta})
\end{equation}

Donde el denominador se corresponde con el desvio total de los $n_1 + n_2$ jugadores.
Recordar que $\Phi$ es la acumulada de la distribuc\'on Normal estandarizada, centrada en 0. 


\subsection{TTT draw-solution (TTT-D)}

\todo[inline]{Entendimos esto, pasarlo}
 
\begin{quotation}
We would thus like to extend the TrueSkill model to incorporate another player-specific parameter which indicates a player's ability to force a draw. 
\end{quotation}

\en{}
\es{Adem\'as de la habilidad genérica, los jugadores pueden tener otras habilidades específicas, como habilidad para forzar un empate.}
%
\en{}
\es{La habilidad de empate, $\varepsilon^t_i$, se modela como Gaussiana}


 
 
 
 
 
 
 
 
 
 
 
 
 
 
 
 
 
 
 
\section{Aproximaci\'on de la posterior (DRAFT)}

La probabilidad de una diferencia cuando se conoce el resultado (Eq.~\ref{eq:p_d}) es una Normal truncada.

\begin{equation}\label{eq:p_d}
\begin{split}
 P(d_1) & \overset{\hfrac{\text{eq}}{\ref{eq:marginal}}}{=}   \prod_{h \in n(d_1)} m_{h \rightarrow d_1} \overset{\hfrac{\text{fig}}{\ref{modelo_trueskill_2vs2}}}{=} m_{f_{d_1} \rightarrow d_1}(d_1) \, m_{f_r \rightarrow d_1}(d_1) \overset{\hfrac{\text{eq}}{\ref{eq:m_fr_d}}}{\underset{\hfrac{\text{eq}}{\ref{eq:m_d_fd}}}{=}}  m_{f_{d_1} \rightarrow d_1}(d_1) \, m_{d_1 \rightarrow f_{d_1}}(d_1)  \\
 & = N(d_1|\delta,\vartheta) \mathbb{I}(d_1 > 0)
\end{split}
\end{equation}


Para tener una posterior normal lo que se hace es un buscar la normal que m\'as se aproxima a esta normal truncada

\vspace{0.3cm}

Se sabe que la Normal que mejor aproxima a una Normal truncada tiene como esperanza

\begin{equation}\label{eq:mean_aprox_double}
 E(X| a < X < b) = \mu + \sigma \frac{N(\frac{a-\mu}{\sigma}) - N(\frac{b-\mu}{\sigma}) }{\Phi(\frac{b-\mu}{\sigma}) - \Phi(\frac{a-\mu}{\sigma}) } = \mu + \sigma \frac{N(\alpha) - N(\beta) }{\Phi(\beta) - \Phi(\alpha) }
\end{equation}

done $\beta = \frac{b-\mu}{\sigma}$ y $\alpha = \frac{a-\mu}{\sigma}$.

Y la varianza 

\begin{equation}\label{eq:variance_aprox_double}
 V(X| a < X < b) = \sigma^2 \Bigg( 1 + \bigg(\frac{\alpha N(\alpha) - \beta N(\beta) }{\Phi(\beta) - \Phi(\alpha) }\bigg) - \bigg(\frac{N(\alpha) - N(\beta) }{\Phi(\beta) - \Phi(\alpha) }\bigg)^2 \Bigg)
\end{equation}

Usando  \'unico truncamiento estas funciones se pueden simplificar como sigue

\begin{equation}\label{eq:mean_aprox_}
\begin{split}
 E(X|  X > a)  & \overset{\hfrac{\text{eq}}{\ref{eq:mean_aprox_double}}}{=}  \mu + \sigma \frac{N(\alpha)}{1 - \Phi(\alpha) } = \mu + \sigma \frac{N(\frac{a-\mu}{\sigma})}{1 - \Phi(\frac{a-\mu}{\sigma}) }\\
 & = \mu + \sigma \underbrace{\frac{N(\frac{\mu-a}{\sigma})}{\Phi(\frac{\mu-a}{\sigma}) }}_{\text{\small $V_1(t)$} } = \mu + \sigma V_1(t)
 \end{split}
\end{equation}

con $t = \frac{\mu -a}{\sigma} = -\alpha  $

\begin{equation}\label{eq:variance_aprox_}
\begin{split}
 V(X|  X > a) & \overset{\hfrac{\text{eq}}{\ref{eq:variance_aprox_double}}}{=} \sigma^2 \Bigg( 1 + \bigg(\frac{\alpha N(\alpha)}{1 - \Phi(\alpha) }\bigg) - \bigg(\frac{N(\alpha)}{1 - \Phi(\alpha) }\bigg)^2 \Bigg) \\
 & = \sigma^2 \Bigg( 1 + \bigg(\frac{\alpha N(-\alpha)}{\Phi(-\alpha) }\bigg) - \bigg(\frac{N(-\alpha)}{\Phi(-\alpha) }\bigg)^2 \Bigg) \\
 & = \sigma^2 \Bigg( 1 + \bigg(\frac{-t N(t)}{\Phi(t) }\bigg) - \bigg(\frac{N(t)}{\Phi(t) }\bigg)^2 \Bigg) \\
 & = \sigma^2 \Big( 1 +  -t V_1(t) - V_1(t)^2 \Big) \\
 & = \sigma^2 \Big( 1 + V_1(t) \big(-t  - V_1(t)\big) \Big)  \\
 & = \sigma^2 \Big( 1 - \underbrace{V_1(t) \big(V_1(t) + t \big)}_{W_1(t)} \Big)  = \sigma^2 \big( 1 - W_1(t) \big) 
 \end{split}
\end{equation}

Luego, la normal aproximada es 

\begin{equation}\label{eq:p*_d}
 \widehat{P}(d_1) \overset{\hfrac{\text{eq}}{\ref{eq:mean_aprox_}}}{\underset{\hfrac{\text{eq}}{\ref{eq:variance_aprox_}}}{=}} N\Bigg(d1 \,  \bigg| \, \underbrace{ \delta + \vartheta \, V_1(t) \,}_{\hfrac{\text{\scriptsize Media}}{\text{\scriptsize aproximada}} \, \text{\small $\widehat{\delta}$}} , \,  \underbrace{ \vartheta^2 \big( 1 - W_1(t) \big) }_{\hfrac{\text{\scriptsize Varianza}}{\text{\scriptsize aproximada}} \,\text{\small $\widehat{\vartheta}^{\,2}$}} \, \Bigg)
\end{equation}

donde, en caso de que no se contemple el empate, $\alpha=\frac{-\delta_1}{\vartheta_1}$.

Teniendo la normal aproximada $\widehat{P}(d_1)$ podemos calcular el mensaje ascendentes aproximado 

\paragraph{$\bm{\widehat{m}_{d \rightarrow f_{d}}(d)}$}

\begin{equation}\label{eq:m^_d_fd}
\begin{split}
 m_{d_1 \rightarrow f_{d_1}}(d_1) &\overset{\hfrac{\text{eq}}{\ref{eq:p_d}}}{=} \frac{P(d_1)}{m_{f_{d_1} \rightarrow d_1}(d_1)} \\
 &\overset{\hfrac{\text{eq}}{\ref{eq:mean_aprox_}}}{\underset{\hfrac{\text{eq}}{\ref{eq:variance_aprox_}}}{\approx}} \frac{\widehat{P}(d_1)}{m_{f_{d_1} \rightarrow d_1}(d_1)} \\
 & \overset{\hfrac{\text{eq}}{\ref{eq:p*_d}}}{\underset{\hfrac{\text{eq}}{\ref{eq:m_fd_d}}}{=}}  \frac{N(d_1 \,  | \,\widehat{\delta} , \, \widehat{\vartheta}^{\,2} )}{N(d_1 | \delta, \vartheta^2)} \\
 &\overset{\hfrac{\text{sec}}{\ref{sec:division_normales}}}{\propto} N(d_1,\delta_{\div},\vartheta_{\div}^2 )
\end{split}
\end{equation}

\begin{equation}
 \vartheta_{\div} = \sqrt{\frac{\widehat{\vartheta}^{\,2}\vartheta^2}{(\vartheta^2 - \widehat{\vartheta}^{\,2})}}
\end{equation}

\begin{equation}
 \delta_{\div} = \frac{(\vartheta^2\widehat{\delta} - \widehat{\vartheta}^{\,2}\delta)}{(\vartheta^2 - \widehat{\vartheta}^{\,2})}
\end{equation}

\paragraph{$\bm{\widehat{m}_{f_d \rightarrow f_{t_a}}(t_a)}$} (Caso ganador)

\begin{equation}\label{eq:^m_fd_ta}
\begin{split}
m_{f_{d_1} \rightarrow t_a}(t_a) & \overset{\hfrac{\text{eq}}{\ref{eq:m_f_v}}}{=} \int \dots \int f_{d_1}(\bm{x}) \prod_{h \in n(f_{d_1}) \setminus \{t_a\} } m_{h \rightarrow f_{d_1}}(h) \, d\bm{x}_{\setminus \{t_a\} }  \\
&\overset{\hfrac{\text{fig}}{\ref{modelo_trueskill_2vs2}}}{\underset{\hfrac{\text{eq}}{\ref{eq:m^_d_fd}}}{\approx}}  \int \dots \int \mathbb{I}(d_1 = t_a - t_b) N(d_1 | \delta_{\div}, \vartheta_{\div}^2) N(t_b | \mu_b , \sigma_b^2 )  \, d\bm{x}_{\setminus \{t_a\} } \\[0.1cm]
\widehat{m}_{f_{d_1} \rightarrow t_a}(t_a)  & \overset{\hfrac{\text{eq}}{\ref{eq:m_f_v}}}{=} \int \int \mathbb{I}(d_1 = t_a - t_b) N(d_1 | \delta_{\div}, \vartheta_{\div}^2) N(t_b | \mu_b , \sigma_b^2 )  \, d{d_1} d_{t_b} \\
& \overset{\hfrac{\text{eq}}{\ref{eq:integral_con_indicadora}}}{=} \int  N(t_a - t_b | \delta_{\div}, \vartheta_{\div}^2) N(t_b | \mu_b , \sigma_b^2 )  \, d_{t_b} \\
& \overset{\hfrac{\text{eq}}{\ref{eq:simetria}}}{=} \int  N( - t_b | \delta_{\div} - t_a, \vartheta_{\div}^2) N(t_b | \mu_b , \sigma_b^2 )  \, d_{t_b} \\
& \overset{\hfrac{\text{eq}}{\ref{eq:simetria}}}{=} \int  N( t_b | t_a - \delta_{\div}, \vartheta_{\div}^2) N(t_b | \mu_b , \sigma_b^2 )  \,  d_{t_b} \\
& \overset{\hfrac{\text{eq}}{\ref{eq:multiplicacion_normales}}}{=} N(t_a - \delta_{\div} \, | \, \mu_b \, , \, \vartheta_{\div}^2 + \sigma_b^2) \\
& \overset{\hfrac{\text{eq}}{\ref{eq:simetria}}}{=} N(t_a \, | \, \mu_b + \delta_{\div} \, , \, \vartheta_{\div}^2 + \sigma_b^2) \\
\end{split}
\end{equation}

\paragraph{$\bm{\widehat{m}_{f_d \rightarrow f_{t_b}}(t_b)}$} (Caso perdedor)

\begin{equation}\label{eq:^m_fd_tb}
\begin{split}
m_{f_{d_1} \rightarrow t_b}(t_b) & \overset{\hfrac{\text{eq}}{\ref{eq:m_f_v}}}{=} \int \dots \int f_{d_1}(\bm{x}) \prod_{h \in n(f_{d_1}) \setminus \{t_b\} } m_{h \rightarrow f_{d_1}}(h) \, d\bm{x}_{\setminus \{t_b\} }  \\
&\overset{\hfrac{\text{fig}}{\ref{modelo_trueskill_2vs2}}}{\underset{\hfrac{\text{eq}}{\ref{eq:m^_d_fd}}}{\approx}}  \int \dots \int \mathbb{I}(d_1 = t_a - t_b) N(d_1 | \delta_{\div}, \vartheta_{\div}^2) N(t_a | \mu_a , \sigma_a^2 )  \, d\bm{x}_{\setminus \{t_a\} } \\[0.1cm]
\widehat{m}_{f_{d_1} \rightarrow t_b}(t_b)  & \overset{\hfrac{\text{eq}}{\ref{eq:m_f_v}}}{=} \int \int \mathbb{I}(d_1 = t_a - t_b) N(d_1 | \delta_{\div}, \vartheta_{\div}^2) N(t_a | \mu_a , \sigma_a^2 )  \, d{d_1} d_{t_a} \\
& \overset{\hfrac{\text{eq}}{\ref{eq:integral_con_indicadora}}}{=} \int  N(t_a - t_b | \delta_{\div}, \vartheta_{\div}^2) N(t_a | \mu_a , \sigma_a^2 )  \, d_{t_a} \\
& \overset{\hfrac{\text{eq}}{\ref{eq:simetria}}}{=} \int  N( t_a | \delta_{\div} + t_b, \vartheta_{\div}^2) N(t_a | \mu_a , \sigma_a^2 )  \, d_{t_a} \\
& \overset{\hfrac{\text{eq}}{\ref{eq:multiplicacion_normales}}}{=} N(t_b + \delta_{\div} \, | \, \mu_a \, , \, \vartheta_{\div}^2 + \sigma_a^2) \\
& \overset{\hfrac{\text{eq}}{\ref{eq:simetria}}}{=} N(t_b \, | \, \mu_a - \delta_{\div} \, , \, \vartheta_{\div}^2 + \sigma_a^2) \\
\end{split}
\end{equation}

\paragraph{$\bm{\widehat{m}_{f_{t_a} \rightarrow p_1}(p_1)}$} (Caso ganador)

\begin{equation}\label{eq:^m_fta_p}
\begin{split}
m_{f_{t_a} \rightarrow p_1}(p_1) & \overset{\hfrac{\text{eq}}{\ref{eq:m_f_v}}}{=} \int \dots \int f_{t_a}(\bm{x}) \prod_{h \in n(f_{t_a}) \setminus \{p_1\} } m_{h \rightarrow f_{t_a}}(h) \, d\bm{x}_{\setminus \{p_1\} }  \\
&\overset{\hfrac{\text{fig}}{\ref{modelo_trueskill_2vs2}}}{\underset{\hfrac{\text{eq}}{\ref{eq:^m_fd_ta}}}{\approx}}  \int \dots \int \mathbb{I}(t_a = p_1 + p_2) N(t_a \, | \, \mu_b + \delta_{\div} \, , \, \vartheta_{\div}^2 + \sigma_b^2) N(p_2 | \mu_2 , \sigma_2^2 + \beta^2)  \, d\bm{x}_{\setminus \{p_1\} } \\[0.1cm]
\widehat{m}_{f_{t_a} \rightarrow p_1}(p_1)  & \overset{\hfrac{\text{eq}}{\ref{eq:m_f_v}}}{=} \int \int \mathbb{I}(t_a = p_1 + p_2) N(t_a \, | \, \mu_b + \delta_{\div} \, , \, \vartheta_{\div}^2 + \sigma_b^2) N(p_2 | \mu_2 , \sigma_2^2 + \beta^2)  \, d{t_a} d_{p_2} \\
& \overset{\hfrac{\text{eq}}{\ref{eq:integral_con_indicadora}}}{=} \int N(p_1 + p_2 \, | \, \mu_b + \delta_{\div} \, , \, \vartheta_{\div}^2 + \sigma_b^2) N(p_2 | \mu_2 , \sigma_2^2+ \beta^2 )   \, d_{p_2} \\
& \overset{\hfrac{\text{eq}}{\ref{eq:simetria}}}{=}\int N(p_2 \, | \, \mu_b - p_1 + \delta_{\div} \, , \, \vartheta_{\div}^2 + \sigma_b^2) N(p_2 | \mu_2 , \sigma_2^2 + \beta^2)   \, d_{p_2} \\
& \overset{\hfrac{\text{eq}}{\ref{eq:multiplicacion_normales}}}{=} N(\mu_b - p_1 + \delta_{\div} \,|\, \mu_2 \,,\,\vartheta_{\div}^2 + \sigma_b^2 + \sigma_2^2 + \beta^2)   \\
& \overset{\hfrac{\text{eq}}{\ref{eq:simetria}}}{=}  N( p_1 \,|\,  (\mu_b - \mu_2) + \delta_{\div}  \,,\,\vartheta_{\div}^2 + \sigma_b^2 + \sigma_2^2 + \beta^2)  \\
\end{split}
\end{equation}


En general
\begin{equation}
\begin{split}
\widehat{m}_{f_{t_a} \rightarrow p_1}(p_1) &= N( p_1 \,|\, (\mu_b  - \sum_{i \in A_a, i\neq1} \mu_i) + \delta_{\div}  \,,\,\vartheta_{\div}^2 + \sigma_b^2 + \sum_{i \in A_a, i\neq1} \sigma_i^2 + \beta^2 ) \\
 & = N( p_1 \,|\, (-\delta + \mu_1) + \delta_{\div}  \,,\,\vartheta_{\div}^2 + (\vartheta^2 - \sigma_1^2 - \beta^2))
\end{split}
\end{equation}


\paragraph{$\bm{\widehat{m}_{f_{t_b} \rightarrow p_3}(p_3)}$} (Caso perdedor)

\begin{equation}\label{eq:^m_ftb_p}
\begin{split}
m_{f_{t_b} \rightarrow p_3}(p_3) & \overset{\hfrac{\text{eq}}{\ref{eq:m_f_v}}}{=} \int \dots \int f_{t_b}(\bm{x}) \prod_{h \in n(f_{t_b}) \setminus \{p_3\} } m_{h \rightarrow f_{t_b}}(h) \, d\bm{x}_{\setminus \{p_3\} }  \\
&\overset{\hfrac{\text{fig}}{\ref{modelo_trueskill_2vs2}}}{\underset{\hfrac{\text{eq}}{\ref{eq:^m_fd_tb}}}{\approx}}  \int \dots \int \mathbb{I}(t_b = p_3 + p_4) N(t_b \, | \, \mu_a - \delta_{\div} \, , \, \vartheta_{\div}^2 + \sigma_a^2) N(p_4 | \mu_4 , \sigma_4^2 + \beta^2)  \, d\bm{x}_{\setminus \{p_3\} } \\[0.1cm]
\widehat{m}_{f_{t_b} \rightarrow p_3}(p_3)  & \overset{\hfrac{\text{eq}}{\ref{eq:m_f_v}}}{=} \int \int \mathbb{I}(t_b = p_3 + p_4) N(t_b \, | \, \mu_a - \delta_{\div} \, , \, \vartheta_{\div}^2 + \sigma_a^2) N(p_4 | \mu_4 , \sigma_4^2 + \beta^2) \, d{t_b} d_{p_4} \\
& \overset{\hfrac{\text{eq}}{\ref{eq:integral_con_indicadora}}}{=} \int N(p_3 + p_4 \, | \, \mu_a - \delta_{\div} \, , \, \vartheta_{\div}^2 + \sigma_a^2) N(p_4 | \mu_4 , \sigma_4^2 + \beta^2) \, d_{p_4} \\
& \overset{\hfrac{\text{eq}}{\ref{eq:simetria}}}{=} \int N(p_4 \, | \, \mu_a - \delta_{\div} - p_3 \, , \, \vartheta_{\div}^2 + \sigma_a^2) N(p_4 | \mu_4 , \sigma_4^2 + \beta^2) \, d_{p_4} \\
& \overset{\hfrac{\text{eq}}{\ref{eq:multiplicacion_normales}}}{=} N(\mu_a - \delta_{\div} - p_3  \,|\, \mu_4 \, , \, \vartheta_{\div}^2 + \sigma_a^2 + \sigma_4^2 + \beta^2)   \\
& \overset{\hfrac{\text{eq}}{\ref{eq:simetria}}}{=}   N(  p_3  \,|\, (\mu_a - \mu_4)  - \delta_{\div} \, , \, \vartheta_{\div}^2 + \sigma_a^2 + \sigma_4^2 + \beta^2)  \\
\end{split}
\end{equation}




\newpage
\section{Posterior exacta multiequipos}

Para calcular la posterior exacta de un equipo $a$ en una partida multiequipos, se puede considerar un factor graph en el que el equipo $a$ es el centro, de modo de poder enviar los mensajes de forma directa.

\begin{figure}[H]
  \centering
  \scalebox{.75}{\tikz{ %
        
      
        \node[factor] (fr) {} ;
        \node[const, above=of fr] (nfr) {$f_r$}; %
	\node[const, above=of nfr] (dfr) {\large $\mathbb{I}(g\,d_{ab}>0)$}; %
        \node[latent, left=of fr] (d) {$d_j$} ; %
        \node[factor, left=of d] (fd) {} ;
        \node[const, above=of fd] (nfd) {$f_d$}; %
        \node[const, above=of nfd] (dfd) {\large $\mathbb{I}(d_{ab}=t_a - t_b)$}; %
        \node[const, below=of d,yshift=-0.15cm] (j) {\footnotesize con signo $g:=1 - 2*\mathbb{I}(o_a < o_b) $};
        
        \node[latent, left=of fd,xshift=-0.9cm] (t) {$t_e$} ; %
        \node[factor, left=of t] (ft) {} ;
        \node[const, above=of ft] (nft) {$f_t$}; %
        \node[const, above=of nft,xshift=0.5cm] (dft) {\large $\mathbb{I}(t_e = \sum_{i \in A_e} p_i)$}; %
        
        \node[latent, left=of ft] (p) {$p_i$} ; %
        \node[factor, left=of p] (fp) {} ;
        \node[const, above=of fp] (nfp) {$f_p$}; %
        \node[const, above=of nfp] (dfp) {\large $N(p_i;s_i,\beta^2)$}; %
 
        \node[latent, left=of fp] (s) {$s_i$} ; %
        \node[factor, left=of s] (fs) {} ;
        \node[const, above=of fs] (nfs) {$f_s$}; %
        \node[const, above=of nfs] (dfs) {\large $N(s_i;\mu_i,\sigma^2)$}; %
         
        \edge[-] {d} {fr};
	\edge[-] {fd} {d};
        \edge[-] {fd} {t};
        \edge[-] {t} {ft};
        \edge[-] {ft} {p};
        \edge[-] {p} {fp};
        \edge[-] {fp} {s};
        \edge[-] {s} {fs};
	
        \plate {personas} {(p)(s)(fs)(nfs)(dfp)(dfs)} {$i \in A_e$}; %
        \node[invisible, below=of ft, yshift=-0.6cm] (inv_below_e) {};
	\node[invisible, above=of ft, yshift=1.1cm] (inv_above_e) {};
	\plate {equipos} {(personas) (t)(ft)(dft) (inv_above_e) (inv_below_e)} {$  \text{con $A$ partici\'on de jugadores }$  \hspace{3cm} $0 < e \leq |A|$}; %
	\node[invisible, below=of fr, yshift=-0.6cm] (inv_below) {};
	\node[invisible, above=of fr, yshift=1.1cm] (inv_above) {};
	\plate {comparaciones} {(fd) (dfd) (d) (fr) (dfr) (inv_below) (inv_above)} {$a$ equipo focal \hspace{0.6cm} $0 < b \leq |A| \ , \ a \neq b$}
	
	
	%\node[const, right= of r, xshift=1.2cm ,yshift=-2.1cm] (result-dist) {$r_{ab} \sim B\left(\Phi\left(\frac{\mu_a - \mu_b}{\sqrt{\beta_a^2+\beta_b^2}}\right)\right)$} ; %
	      
        }
}
  \caption{\small Grafo de factorizaci\'on del modelo \texttt{TrueSkill} alternativo, en el que el equipo un focal $a$ se encuentra en el centro.}
  \label{modelo_trueskill_2vs2}
\end{figure}



\paragraph{$\bm{m_{t_a \rightarrow f_{t_a}}(t_a)}:$}

\begin{equation}\label{eq:m_ta_ft}
\begin{split}
 m_{t_a \rightarrow f_{t_a}}(t_a) \overset{\hfrac{\text{eq}}{\ref{eq:m_v_f}}}{=} \prod_{g \in n(t_a) \setminus  \{f_{t_a} \}} m_{g \rightarrow t_a} (t_a)  \overset{\hfrac{\text{eq}}{\ref{eq:m_fd_ta}}}{=} \Phi(t_a|\mu_b,\sigma_b^2) \overset{\hfrac{\mu_b}{\sigma_b}}{=} \Phi \Big(t_a| \sum_{i \in A_b} \mu_i , \sum_{i \in A_b} \beta^2 + \sigma_i^2 \Big) 
\end{split}
\end{equation}

\section{Full TrueSkill}

\begin{center}
\tikz{ %        
        \node[factor, xshift=-5cm] (fta) {} ;
        \node[const, right=of fta] (nfta) {$f_{t_0}$}; %
        \node[latent, below=of fta,yshift=-0.5cm] (ta) {$t_0$} ; %
        
        \node[factor] (ftb) {} ;
        \node[const, right=of ftb] (nftb) {$f_{t_1}$}; %
        \node[latent, below=of ftb,yshift=-0.5cm] (tb) {$t_1$} ; %
        
        \node[factor, xshift=5cm] (ftc) {} ;
        \node[const, right=of ftc] (nftc) {$f_{t_2}$}; %        
        \node[latent, below=of ftc,yshift=-0.5cm] (tc) {$t_2$} ; %
        
        \node[factor, below=of tb, xshift=-3cm] (fd1) {} ;
        \node[const, left=of fd1] (nfd1) {$f_{d_0}$}; %        
        \node[latent, below=of fd1,yshift=-1cm] (d1) {$d_{0}$} ; %
        \node[factor, below=of d1,yshift=-1cm] (fr1) {} ;
        
        \node[factor, below=of tb, xshift=3cm] (fd2) {} ;
        \node[const, above=of fd2] (nfd2) {$f_{d_{1}}$}; %        
        \node[latent, below=of fd2,yshift=-1cm] (d2) {$d_{1}$} ; %
        \node[factor, below=of d2,yshift=-1cm] (fr2) {} ;
        
        \edge[-] {ta} {fta,fd1}
        \edge[-] {tb} {ftb,fd1,fd2}
        \edge[-] {tc} {ftc,fd2}
        \edge[-] {d1} {fd1,fr1}
        \edge[-] {d2} {fd2,fr2}
        
        \path[draw, -latex, fill=black!50,sloped] (fd1) edge[bend left,draw=black!50] node[midway,above,color=black!75] {\scriptsize  \emph{loseLikelihood}$(t_1)$} (tb);
        
        \path[draw, -latex, fill=black!50,sloped] (tb) edge[bend left,draw=black!50] node[midway,below,color=black!75] {\scriptsize  \emph{losePosterior}$(t_1)$} (fd1);
        
        \path[draw, -latex, fill=black!50,sloped] (fd2) edge[bend right,draw=black!50] node[midway,above,color=black!75] {\scriptsize \emph{winLikelihood}$(t_1)$} (tb);
        
        \path[draw, -latex, fill=black!50,sloped] (tb) edge[bend right,draw=black!50] node[midway,below,color=black!75] {\scriptsize \emph{winPosterior}$(t_1)$} (fd2);
        
        \path[draw, -latex, fill=black!50,sloped] (fta) edge[bend left,draw=black!50] node[midway,above,color=black!75] {\scriptsize \emph{prior}$(t_0)$} (ta);
        
        \path[draw, -latex, fill=black!50,sloped] (fr1) edge[bend left,draw=black!50] node[midway,above,color=black!75, rotate=180] {\scriptsize \emph{likelihood}$(d_{0})$} (d1);
        
        \path[draw, -latex, fill=black!50,sloped] (d1) edge[bend right,draw=black!50] node[midway,above,color=black!75] {\scriptsize \emph{\ \ posterior}$(d_{0})$} (fd1);
        
        \path[draw, -latex, fill=black!50,sloped] (fd1) edge[bend right,draw=black!50] node[midway,above,color=black!75] {\scriptsize \emph{prior}$(d_{0})$} (d1);
        
        \path[draw, -latex, fill=black!50,sloped] (tc) edge[bend left,draw=black!50] node[midway,above,color=black!75] {\scriptsize \emph{posterior}$(t_0)$} (ftc);
        
        
        %\path[draw, -latex, fill=black!50,sloped] (fr2) edge[bend left,draw=black!50] node[midway,above,color=black!75, rotate=180] {\scriptsize \textbf{5:} \emph{likelihood}$(d_{0})$} (d2);
        
        %\path[draw, -latex, fill=black!50,sloped] (fd2) edge[bend left,draw=black!50] node[midway,above,color=black!75] {\scriptsize \textbf{4:} \emph{prior}$(d_{0})$} (d2);
        
        
} 
\end{center}

\begin{equation}
 p(t_e) = \emph{prior}(t_e) \cdot \emph{looseLikelihood}(t_e) \cdot \emph{winLikelihood}(t_e)
\end{equation}

\begin{equation}
 p(d_i) = \emph{prior}(d_i) \cdot \emph{likelihood}(d_i)
\end{equation}

There is a mutual dependency between the team performance (difference) marginals, $p(d_i)$.
That makes impossible to perform a one shot inference.
The basic idea is to update repeatedly forward and backward all messages in the shortest path between any two marginals $p(d_i)$ until convergence.
The messages that are not yet defined, for example the \emph{winLikelihood}$(t_1)$ in the first forward pass, must be replace it by a neutral form, like a unit scalar or a Gaussian distribution with infinite variance.
If the messages have already been defined, then use the last available definition.

\paragraph{$\bm{\emph{winPosterior}(t_i)}$:}

\begin{equation}
 \begin{split}
  \emph{winPosterior}(t_{i}) = \emph{prior}(t_i) \cdot \emph{winLikelihood}(t_i)
 \end{split}
\end{equation}


\paragraph{$\bm{\emph{losePosterior}(t_i)}$:}

\begin{equation}
 \begin{split}
  \emph{losePosterior}(t_{i}) = \emph{prior}(t_i) \cdot \emph{loseLikelihood}(t_i)
 \end{split}
\end{equation}


\paragraph{$\bm{\emph{prior}(d_i)}$:}

\begin{equation}
 \begin{split}
  \emph{prior}(d_{i}) = \iint \mathbb{I}(d_{i} = t_i - t_{i+1} ) \cdot winPosterior(t_i) \cdot losePosterior(t_{i+1}) \, dt_idt_{i+1} 
 \end{split}
\end{equation}


\paragraph{$\bm{\widehat{\emph{likelihood}}(d_i)}$:}

\begin{equation}
\widehat{p}(d_i) = \emph{prior}(d_i) \, \widehat{\emph{likelihood}}(d_i)
\end{equation}

Following the expectation propagation algorithm, we approximate the \emph{likelihood}$(d_i)$ by approximating the marginal $p(d_i)$ via moment matching, resulting in $\widehat{p}(d_i)$ a Gaussian with the same mean and variance as $p(d_i)$.
\begin{equation}
\begin{split}
 \widehat{p}(d_i) & = \emph{approx}(p(d_i)) = \emph{approx}(\emph{prior}(d_i) \cdot \emph{likelihood}(d_i)) 
 \end{split}
\end{equation}

Then, 
\begin{equation}
\begin{split}
\widehat{\emph{likelihood}}(d_i) &= \frac{\emph{approx}(\emph{prior}(d_i) \cdot \emph{likelihood}(d_i))}{ \emph{prior}(d_i)}
\end{split}
\end{equation}

\paragraph{$\bm{\widehat{\emph{posterior}}(d_i)}$:}


\begin{equation}
\begin{split}
\widehat{\emph{posterior}}(d_i) &= \widehat{\emph{likelihood}}(d_i)
\end{split}
\end{equation}

\paragraph{\emph{loseLikelihood}$\bm(t_{i+1})$:}

\begin{equation}
 \begin{split}
  \emph{loseLikelihood}(t_{i+1}) & = \iint \mathbb{I}(d_{i} = t_i - t_{i+1} ) \emph{winPosterior}(t_i) \widehat{\emph{posterior}}(d_i) \, dd_idt_{i+1} 
  \end{split}
\end{equation}


\paragraph{\emph{winLikelihood}$\bm(t_{i})$:}

\begin{equation}
 \begin{split}
  \emph{winLikelihood}(t_{i}) = \iint \mathbb{I}(d_{i} = t_i - t_{i+1} ) \emph{losePosterior}(t_{i+1}) \widehat{\emph{posterior}}(d_i) \, dd_idt_{i+1} 
 \end{split}
\end{equation}




\section{TrueSkill through Time (TTT)}

Consider a game in which $N$ players $\{1, \dots, N\}$ are compete over a period of $T$ time steps or temporal batch (e.g. day, week, month, year).
Denote the series of games outcomes between two players $i$ and $j$ within a temporal batch $t$ by $\vm{y}^t_{ij}(k)$, where $k \in \{1,\dots,K^t_{ij}\}$, with $K^t_{ij}$ the number of games outcomes available for that pair of players in that temporal batch.

\begin{mdframed}[backgroundcolor=black!15]
 Note that the following conclusions are based on the idea that original TrueSkill \emph{porceeds through the years (temporal batch) in order, but (within a temporal batch) goes through games outcome in random order}.
 This is not true. 
 Original TrueSkill algorithm does not need any temporal batch for inference.
 So neither goes through games outcome in random order within a year or any other temporal batch.
 By contrast, original TrueSkill sorts games before the inference stage by their game date.
 Sometime it is true, specially in in online game, that we can find users who play more than one game at the same time.
 In this cases it is not clear how to sort those games.
 However, this situation never holds more than a few hours.
 The unsorted games are an exception rather than a rule in original TrueSkill.
 \textbf{So why the authors of TTT adopt temporal batching, unnecessary for the orginal TrueSkill?}
 \textbf{There is a hidden reason for adopting temporal batching, only necessary for TTT?}
 There is no problem if we add new features to an algorithm.
 But in this case, the adoption of this new feature is badly justified.
\end{mdframed}

\begin{itemize}
 \item Inference within a temporal batch (day, week, month) depends on the random order chosen for the update. However the results should be independent of the order of games within the temporal batch.
 \item Information across temporal batch is only propagated forward in time, so we can not propagate information backwards.
\end{itemize}

\begin{mdframed}[backgroundcolor=black!15] 
 What happens if we define very small and very large temporal batch. 
 For example consider the follow three games in temporal order: A beats B, then B beats C and finally C beats A.
 If we define \textbf{short temporal batch} such that only cover one game at a time, then there will be no within convergence step for which the results of inference should be independent of the order of games.
 If we define \textbf{long temporal batch} such that cover all games, then there will be no time steps to propagate information backwards.
\end{mdframed}

The neighbor factor graph of a temporal skill variable $s_i^t$, is

\begin{center}
\tikz{ %        
        \node[latent] (s0) {$s_i^{t-1}$} ; %
        
        \node[factor, right=of s0,xshift=1cm ] (fs1) {} ;
        \node[const, above=of fs1] (nfs1) {$f_{s_i^{t}}$}; %
        
        \node[latent, right=of fs1, xshift=1.25cm] (s1) {$s_i^t$} ; %
        
        \node[factor, right=of s1, xshift=1.25cm ] (fs2) {} ;
        \node[const, above=of fs2] (nfs2) {$f_{s_i^{t+1}}$}; %
        
        \node[latent, right=of fs2,xshift=1cm] (s2) {$s_i^{t+1}$} ; %
        
        \node[factor, below=of s1,xshift=-1.4cm,yshift=-1cm] (fp0) {} ;
        \node[const, right=of fp0] (nfp0) {$f_{p_i^{t}(1)}$}; %
        
        \node[factor, color=white, below=of s1] (fp1) {} ;
        %\node[const, right=of fp1] (nfp1) {$f_{p_i^{t}(2)}$}; %
        
        \node[factor, below=of s1,xshift=1.4cm,yshift=-1cm] (fp2) {} ;
        \node[const, left=of fp2] (nfp2) {$f_{p_i^{t}(k)}$}; %
        
        \node[latent, below=of fp0] (p0) {\footnotesize$p_i^{t}(1)$} ; %
        %\node[latent, below=of fp1] (p1) {\footnotesize$p_i^{t}(2)$} ; %
        \node[latent, below=of fp2] (p2) {\footnotesize$p_i^{t}(k)$} ; %
        
        %\draw[bend right=90] (fs1) arc (s1) node[midway,above]{label};
        %\draw[bend left,->]  (fs1) to node [auto] {Link} (s1);
        \edge[-] {s1} {fp0,fp1,fp2};
        \edge[-] {fp0} {p0};
        %\edge[-] {fp1} {p1};
        \edge[-] {fp2} {p2};
        \edge[-] {fs1} {s0,s1};
        \edge[-] {fs2} {s1,s2};
        %\edge[bend right] {s0} {fs1};
        \path[draw, -latex, fill=black!50] (s0) edge[bend right,draw=black!50] node[midway,below,color=black!75] {\scriptsize \emph{posterior}$(t-1)$} (fs1);
        \path[draw, -latex, fill=black!50] (fs1) edge[bend left,draw=black!50] node[midway,above,color=black!75] {\scriptsize \emph{prior}$(t)$} (s1);
        \path[draw, -latex, fill=black!50] (s2) edge[bend left,draw=black!50] node[midway,below,color=black!75] {\scriptsize \emph{\ \ inversePosterior}$(t+1)$} (fs2);
        \path[draw, -latex, fill=black!50] (fs2) edge[bend right,draw=black!50] node[midway,above,color=black!75] {\scriptsize \emph{inversePrior}$(t)$} (s1);
        \path[draw, -latex, fill=black!50,sloped] (fp0) edge[bend left,draw=black!50] node[midway,above,color=black!75] {\scriptsize \emph{likelihood}$(t,k)$} (s1);
        \path[draw, -latex, fill=black!50,sloped] (s1) edge[bend left,draw=black!50] node[midway,above,color=black!75] {\scriptsize \emph{\ \ withinPrior}$(t,k)$} (fp2);
} 
\end{center}

The arrows represents messages computed by the sum-product algorithm.
The names were selected for the sole purpose of simplifying the notation.
By the sum-product algorithm we know that the marginal distribution any variable is the product of the messages the variable receives from its neighbors,

\begin{equation}
 p(s^t) = \prod_{h \in n(s^t)} m_{h \rightarrow s^t}(s^t)
\end{equation}

Replaced the messages by the selected names, this distribution can be expressed as, 

\begin{equation}
 p(s_i^t) = \emph{prior}_i(t) \cdot \emph{inversePrior}_i(t) \cdot \prod_{k=1}^{K_i^t} \emph{likelihood}_i(t,k)
\end{equation}

There is a mutual dependency between forward and backward messages that make imposible a one shot inference iteration.
The basic idea is to update repeatedly forward and backward until convergence, making sure that the effect of the previous update is removed before the new effect is added.
The messages that are not yet defined, for example the inverse prior in the first forward pass, are replace it with by a neutral form like a unit scalar or a Gaussian distribution with infinite variance.

\vspace{0.3cm}

\paragraph{\emph{posterior}$\bm{_i(t-1)}$:}

\begin{equation}
\begin{split}
  \emph{posterior}_i(t-1)&= m_{s_i^{t-1}\rightarrow f_{s_i^t}}(s_i^{t-1}) \\ 
 &= \emph{prior}_i(t-1) \cdot \prod_{k=1}^{K_i^{t-1}} \emph{likelihood}_i(t-1,k)\\
 &= p(s_i^{t-1}) \, \emph{inversePrior}_i(t-1)^{-1}
 \end{split}
\end{equation}

 
 \paragraph{\emph{prior}$_i\bm{(t)}$:}
 
 \begin{equation}
 \begin{split}
  \emph{prior}_i(t) &= m_{f_{s_i^t} \rightarrow s_i^t}(s_i^t) \\
  &= \int f_{s^t} \cdot \emph{posterior}_i(t-1) \, ds_i^{t-1}
  \end{split}
 \end{equation}

 \paragraph{\emph{likelihood}$_i\bm{(t,k)}$:}
 
 \begin{equation}
 \begin{split}
  \emph{likelihood}_i(t,k) &= m_{f_{p^{t}(k)} \rightarrow s^t}(s^t) \\
  &= \emph{trueSkillVariationalLikelihood}_i(t,k)
  \end{split}
 \end{equation}
 
 \paragraph{\emph{withinPrior}$_i\bm{(t,k)}$:}

 
 \begin{equation}
 \begin{split}
 \emph{withinPrior}_i(t,k) &= m_{s_i^t \rightarrow f_{p_i^t(k)}}(s_i^t)  \\
 &= \emph{prior}_i(t) \cdot \emph{inversePrior}_i(t) \cdot \prod_{q=1, q\neq k}^{K_i^t} \emph{likelihood}_i(t,q) \\
 &= p(s_i^t) \cdot \emph{likelihood}_i(t,k)^{-1}
  \end{split}
 \end{equation}
 
 \paragraph{\emph{inversePrior}$\bm{_i(t)}$: }
 
 \begin{equation}
 \begin{split}
 \emph{inversePrior}_i(t) &= m_{f_{s_i^{t+1}} \rightarrow s_i^t}(s_i^t) \\
 &= \int f_{s_i^{t+1}} \cdot \emph{inversePosterior}_i(t+1) \, ds_i^{t+1}
 \end{split}
 \end{equation}
  
  \paragraph{\emph{inversePosterior}$\bm{_i(t+1)}$: }
  
 \begin{equation}
 \begin{split}
 \emph{inversePosterior}_i(t+1) &= m_{s_i^{t+1} \rightarrow f_{s_i^{t+1}}}(s_i^{t+1}) \\
 &= \emph{inversePrior}_i(t+1) \cdot \prod_{k=1}^{K_i^{t+1}} \emph{likelihood}_i(t+1,k) \\
 &= p(s_i^{t+1}) \cdot \emph{prior}_i(t+1)^{-1}
 \end{split}
 \end{equation}

\subsection{Convergence}

\paragraph{Temporal batch} Within a temporal batch $t$, we go through the games outcomes $\vm{y}^t$ several times until convergence.
The update for a game outcome $y^t_i(k)$ is performed by the same way as original TrueSkill using as prior the \emph{withinPrior}$_i(t,k)$ and saving the \emph{likelihood}$_i(t,k)$.

\paragraph{Forward-Backward Propagation}
At each forward pass we store each forward message, i.e. \emph{prior}$_i(t+1)$.
And at each backward pass we compute the backward message, i.e. \emph{inversePrior}$_i(t-1)$.
This messages, together with the last stored \emph{likelihood}$_i(t,k)$, will be used to compute each \emph{withinPrior}$_i(t,k)$ at the temporal batch.





%%%%%%%%%%%%%%%%%%%%%%%%%%%%%%%%%%%%%%%%%%%%
\newpage

{\scriptsize
\bibliographystyle{../../../licar/bibliografia/Gaming/plos2015}
\bibliography{../../../licar/bibliografia/Gaming/gaming.bib}
}

%%%%%%%%%%%%%%%%%%%%%%%%%%%%%%%%%%%































































































\newpage
\section{Anexo A. Propiedades de las funciones de densidad Normales}

\subsection{Multiplicaci\'on de normales}\label{multiplicacion_normales}

Luego, el problema que tenemos que resolver es
\begin{equation}
 \int N(x;\mu_1,\sigma_1^2)N(x;\mu_2,\sigma_2^2) dx
\end{equation}

Por defnici\'on,
\begin{equation}
\begin{split}
 N(x;y,\beta^2)N(x;\mu,\sigma^2) & = \frac{1}{\sqrt{2\pi}\sigma_1}e^{-\frac{(x-\mu_1)^2}{2\sigma_1^2}} \frac{1}{\sqrt{2\pi}\sigma_2}e^{-\frac{(x-\mu_2)^2}{2\sigma_2^2}}  \\
 & = \frac{1}{2\pi\sigma_1\sigma_2}\text{exp}\Bigg(-\underbrace{\left( \frac{(x-\mu_1)^2}{2\sigma_1^2} + \frac{(x-\mu_2)^2}{2\sigma_2^2} \right)}_{\theta} \Bigg)
\end{split}
\end{equation}

Luego,
\begin{equation}
 \theta = \frac{\sigma_2^2(x^2 + \mu_1^2 - 2x\mu_1) + \sigma_1^2(x^2 + \mu_2^2 - 2x\mu_2) }{2\sigma_1^2\sigma_2^2}
\end{equation}

Expando y reordeno los factores por potencias de $x$
\begin{equation}
 \frac{(\sigma_1^2 + \sigma_2^2) x^2 - (2\mu_1\sigma_2^2 + 2\mu_2\sigma_1^2) x + (\mu_1^2\sigma_2^2 + \mu_2^2\sigma_1^2)}{2\sigma_1^2\sigma_2^2}
\end{equation}

Divido al numerador y el denominador por el factor de $x^2$
\begin{equation}
 \frac{x^2 - 2\frac{(\mu_1\sigma_2^2 + \mu_2\sigma_1^2)}{(\sigma_1^2 + \sigma_2^2) } x + \frac{(\mu_1^2\sigma_2^2 + \mu_2^2\sigma_1^2)}{(\sigma_1^2 + \sigma_2^2) }}{2\frac{\sigma_1^2\sigma_2^2}{(\sigma_1^2 + \sigma_2^2)}}
\end{equation}

Esta ecuaci\'on es cuadr\'atica en x, y por lo tanto es proporcional a una funci\'on de densidad gausiana con desv\'io
\begin{equation}
\sigma_{\times} = \sqrt{\frac{\sigma_1^2\sigma_2^2}{\sigma_1^2+\sigma_2^2}}  
\end{equation}

y media
\begin{equation}
 \mu_{\times} = \frac{(\mu_1\sigma_2^2 + \mu_2\sigma_1^2)}{(\sigma_1^2 + \sigma_2^2) }
\end{equation}

Dado que un t\'ermino $\varepsilon = 0$ puede ser agregado para completar el cuadrado en $\theta$, esta prueba es suficiente cuando no se necesita una normalizaci\'on.
Sea, 
\begin{equation}
 \varepsilon = \frac{\mu_{\times}^2-\mu_{\times}^2}{2\sigma_{\times}^2} = 0
\end{equation}

Al agregar este t\'ermino a $\theta$ tenemos
\begin{equation}
 \theta = \frac{x^2 - 2\mu_{\times}x + \mu_{\times}^2 }{2\sigma_{\times}^2} + \underbrace{\frac{ \frac{(\mu_1^2\sigma_2^2 + \mu_2^2\sigma_1^2)}{(\sigma_1^2 + \sigma_2^2) } - \mu_{\times}^2}{2\sigma_{\times}^2}}_{\varphi}
\end{equation}

Reorganizando el t\'ermino $\varphi$
\begin{equation}
\begin{split}
\varphi & = \frac{\frac{(\mu_1^2\sigma_2^2 + \mu_2^2\sigma_1^2)}{(\sigma_1^2 + \sigma_2^2) } - \left(\frac{(\mu_1\sigma_2^2 + \mu_2\sigma_1^2)}{(\sigma_1^2 + \sigma_2^2) }\right)^2 }{2\frac{\sigma_1^2\sigma_2^2}{\sigma_1^2+\sigma_2^2}}  \\
& = \frac{(\sigma_1^2 + \sigma_2^2)(\mu_1^2\sigma_2^2 + \mu_2^2\sigma_1^2) - (\mu_1\sigma_2^2 + \mu_2\sigma_1^2)^2}{\sigma_1^2 + \sigma_2^2}\frac{1}{2\sigma_1^2\sigma_2^2} \\[0.3cm]
& = \frac{(\mu_1^2\sigma_1^2\sigma_2^2 + \cancel{\mu_2^2\sigma_1^4} + \bcancel{\mu_1^2\sigma_2^4} + \mu_2^2\sigma_1^2\sigma_2^2) - (\bcancel{\mu_1^2\sigma_2^4} + 2\mu_1\mu_2\sigma_1^2\sigma_2^2 + \cancel{\mu_2^2\sigma_1^4} )}{\sigma_1^2 + \sigma_2^2}  \frac{1}{2\sigma_1^2\sigma_2^2} \\[0.3cm] 
& = \frac{(\sigma_1^2\sigma_2^2)(\mu_1^2 + \mu_2^2 - 2\mu_1\mu_2)}{\sigma_1^2 + \sigma_2^2}\frac{1}{2\sigma_1^2\sigma_2^2} = \frac{\mu_1^2 + \mu_2^2 - 2\mu_1\mu_2}{2(\sigma_1^2 + \sigma_2^2)} = \frac{(\mu_1 - \mu_2)^2}{2(\sigma_1^2 + \sigma_2^2)}
\end{split}
\end{equation}

Luego,
\begin{equation}
 \theta = \frac{(x-\mu_{\times})^2}{2\sigma_{\times}^2} + \frac{(\mu_1 - \mu_2)^2}{2(\sigma_1^2 + \sigma_2^2)} 
\end{equation}

Colocando esta forma de $\theta$ en su lugar
\begin{equation}
\begin{split}
 N(x;y,\beta^2)N(x;\mu,\sigma^2) & = \frac{1}{2\pi\sigma_1\sigma_2}\text{exp}\Bigg(-\underbrace{\left( \frac{(x-\mu_{\times})^2}{2\sigma_{\times}^2} + \frac{(\mu_1 - \mu_2)^2}{2(\sigma_1^2 + \sigma_2^2)} \right)}_{\theta} \Bigg) \\
 & = \frac{1}{2\pi\sigma_1\sigma_2}\text{exp}\left(  - \frac{(x-\mu_{\times})^2}{2\sigma_{\times}^2} \right) \text{exp} \left( - \frac{(\mu_1 - \mu_2)^2}{2(\sigma_1^2 + \sigma_2^2)} \right) 
\end{split}
\end{equation}

Multiplicando por $\sigma_{\times}\sigma_{\times}^{-1}$
\begin{equation}
\overbrace{\frac{\cancel{\sigma_1\sigma_2}}{\sqrt{\sigma_1^2+\sigma_2^2}}}^{\sigma_{\times}} \frac{1}{\sigma_{\times}} \frac{1}{2\pi\cancel{\sigma_1\sigma_2}}\text{exp}\left(  - \frac{(x-\mu_{\times})^2}{2\sigma_{\times}^2} \right) \text{exp} \left( - \frac{(\mu_1 - \mu_2)^2}{2(\sigma_1^2 + \sigma_2^2)} \right)
\end{equation}

Luego,
\begin{equation}
 \frac{1}{\sqrt{2\pi}\sigma_{\times}}\text{exp}\left(  - \frac{(x-\mu_{\times})^2}{2\sigma_{\times}^2} \right) \frac{1}{\sqrt{2\pi(\sigma_1^2+\sigma_2^2)}} \text{exp} \left( - \frac{(\mu_1 - \mu_2)^2}{2(\sigma_1^2 + \sigma_2^2)} \right)
\end{equation}

Retonando a la integral
\begin{equation}
\begin{split}
I & = \int N(x;\mu_{\times},\sigma_{\times}^2) \overbrace{N(\mu_1;\mu_2,\sigma_1^2 + \sigma_2^2)}^{\text{Escalar independiente de x}} dx \\[0.3cm]
& = N(\mu_1;\mu_2,\sigma_1^2 + \sigma_2^2) \underbrace{\int N(x,\mu_{\times},\sigma_{\times}^2)  dx}_{\text{Integra 1}} \\
& = N(\mu_1;\mu_2,\sigma_1^2 + \sigma_2^2)
\end{split}
\end{equation}

\subsection{Suma de n normales}\label{suma_normales_induccion}

Sabemos que 

\begin{equation}
t_n = \sum_{i=1}^n x_i \sim \int \dots \int \mathbb{I}(t_n= \sum_{i=1}^n x_i ) \left( \prod_{i=1}^n N(x_i;\mu_i,\sigma_i^2) \right) dx_1 \dots dx_n = N(t;\sum_{i=1}^n \mu_i,\sum_{i=1}^n \sigma_i^2 ) 
\end{equation}


Queremos probar por inducci\'on.
\begin{equation}
 P(n):= \int \dots \int \mathbb{I}(t_n= \sum_{i=1}^n x_i ) \left( \prod_{i=1}^n N(x_i;\mu_i,\sigma_i^2) \right) dx_1 \dots dx_n \overset{?}{=} N(t;\sum_{i=1}^n \mu_i,\sum_{i=1}^n \sigma_i^2 )
\end{equation}

\paragraph{Casos base} 

\begin{equation}
\begin{split}
 P(1) := \int \mathbb{I}(t_1 = x_1) N(x_1;\mu_1,\sigma_1^2) dx_1 = N(x;\mu_1,\sigma_1^2)
\end{split}
\end{equation}

Luego $P(1)$ es verdadera.

\begin{equation}
 \begin{split}
P(2) & := \iint \mathbb{I}(t_2 = x_1 + x_2) N(x_1|\mu_1, \sigma_1^2)N(x_2|\mu_2, \sigma_2^2) dx_1dx_2 \\
 &= \int N(x_1|\mu_1, \sigma_1^2) N(t_2 - x_1|\mu_2, \sigma_2^2) dx_1   \\
 & = \int N(x_1|\mu_1, \sigma_1^2) N(x_1|t_2 - \mu_2, \sigma_2^2) dx_1 \\
 & \overset{*}{=} \int \underbrace{N(t_2|\mu_1+\mu_2,\sigma_1^2 + \sigma_2^2)}_{\text{const.}} \underbrace{N(x_1|\mu_{*},\sigma_{*}^2) dx_1}_{1} \\
 & = N(t_2|\mu_1+\mu_2,\sigma_1^2 + \sigma_2^2)
 \end{split}
 \end{equation}
 
 Donde $\overset{*}{=}$ vale por la demostraci\'on de miltiplicaci\'on de normales en la secci\'on~\ref{multiplicacion_normales}. 
 Luego, vale $P(2)$.

 
\paragraph{Paso inductivo} $P(n) \Rightarrow P(n+1)$

Sea,
\begin{equation}
 P(n) :=\int \dots \int \mathbb{I}(t_n= \sum_{i=1}^n x_i ) \left( \prod_{i=1}^n N(x_i;\mu_i,\sigma_i^2) \right) dx_1 \dots dx_n = N(t;\sum_{i=1}^n \mu_i,\sum_{i=1}^n \sigma_i^2 ) 
\end{equation}

Queremos ver que vale $P(n+1)$

\begin{equation}
 P(n+1) := \int \dots \int \mathbb{I}(t_{n+1}=  x_{n+1} + \sum_{i=1}^{n} x_i ) \left( \prod_{i=1}^{n} N(x_i;\mu_i,\sigma_i^2) \right) N(x_{n+1};\mu_{n+1},\sigma_{n+1}^2) dx_1 \dots dx_{n} dx_{n+1}  
\end{equation}

Por independencia
\begin{equation}
 \int N(x_{n+1};\mu_{n+1},\sigma_{n+1}^2) \left( \int \dots \int \mathbb{I}(t_{n+1}= x_{n+1} + \sum_{i=1}^{n} x_i ) \left( \prod_{i=1}^{n} N(x_i;\mu_i,\sigma_i^2) \right)  dx_1 \dots dx_{n}\right) dx_{n+1}  
\end{equation}

Por hip\'otesis inductiva
\begin{equation}
 \int N(x_{n+1};\mu_{n+1},\sigma_{n+1}^2) N(t-x_{n+1};\sum_{i=1}^n \mu_i,\sum_{i=1}^n \sigma_i^2) dx_{n+1}  
\end{equation}

Por demostraci\'on de la secci\'on~\ref{multiplicacion_normales},
\begin{equation}
  N(t;\mu_{n+1}+\sum_{i=1}^{n} \mu_i,\sigma_{n+1}^2 \sum_{i=1}^n \sigma_i^2) dx_{n+1}  
\end{equation}

Luego, vale $P(n+1)$.

\subsection{Normal por acumulada de Normal}

Queremos resolver la integral

\begin{equation}
 f(x) = \int N(y;\mu_1,\sigma_1^2)\Phi(y+x;\mu_2,\sigma_2^2) dy
\end{equation}

Para ello trabajamos con la drivada $\frac{\partial}{\partial x}f(x) = \theta(x)$,
\begin{equation}
 \theta(x) = \frac{\partial}{\partial x}\int N(y;\mu_1,\sigma_1^2)\Phi(y+x;\mu_2,\sigma_2^2) dy
\end{equation}

Por ``Dominated convergence theorem, integrales y derivadas pueden intercambiar posiciones.
\begin{equation}
 \theta(x) = \int N(y;\mu_1,\sigma_1^2)\frac{\partial}{\partial x}\Phi(y+x;\mu_2,\sigma_2^2) dy
\end{equation}

La derivada de $\Phi$ es justamente una normal,
\begin{equation}
\begin{split}
\theta(x) & = \int N(y;\mu_1,\sigma_1^2)N(y+x;\mu_2,\sigma_2^2) dy \\ 
& = \int N(y;\mu_1,\sigma_1^2)N(y;\mu_2-x,\sigma_2^2) dy 
\end{split}
\end{equation}

Por la demostraci\'on de la secci\'on~\ref{multiplicacion_normales} sabemos
\begin{equation}
 \theta(x) = N(\mu_1; \mu_2 - x, \sigma_1^2 + \sigma_2^2)
\end{equation}

Por simetr\'ia
\begin{equation}
 \theta(x) = N(x; \mu_2 - \mu_1, \sigma_1^2 + \sigma_2^2)
\end{equation}

Retornando a $f(x)$
\begin{equation}
 f(x) = \Phi(x; \mu_2 - \mu_1, \sigma_1^2 + \sigma_2^2)
\end{equation}

\subsection{Divisi\'on de Normales}\label{sec:division_normales}

\begin{equation}
\kappa = \frac{N(x;\mu_f,\sigma_f^2)}{N(x;\mu_g,\sigma_g^2)} = N(x;\mu_f,\sigma_f^2)N(x;\mu_g,\sigma_g^2)^{-1}
\end{equation}

Por definici\'on
\begin{equation}
\begin{split}
\kappa & = \frac{1}{\sqrt{2\pi}\sigma_f}e^{-\left(\frac{(x-\mu_f)^2}{2\sigma_f^2}\right)} \left( \frac{1}{\sqrt{2\pi}\sigma_g}e^{-\left(\frac{(x-\mu_g)^2}{2\sigma_g^2}\right)} \right)^{-1} \\[0.3cm]
& = \frac{1}{\cancel{\sqrt{2\pi}}\sigma_f}e^{-\left(\frac{(x-\mu_f)^2}{2\sigma_f^2}\right)} \frac{\cancel{\sqrt{2\pi}}\sigma_g}{1} e^{\left(\frac{(x-\mu_g)^2}{2\sigma_g^2}\right)} \\[0.3cm]
& = \frac{\sigma_g}{\sigma_f}\text{exp}\Bigg(-\underbrace{\Big(\frac{(x-\mu_f)^2}{2\sigma_f^2} - \frac{(x-\mu_g)^2}{2\sigma_g^2}\Big)}_{\theta}\Bigg)
\end{split}
\end{equation}

Reorganizando $\theta$
\begin{equation}
\begin{split}
 \theta & = \frac{(x-\mu_f)^2}{2\sigma_f^2} - \frac{(x-\mu_g)^2}{2\sigma_g^2} = \frac{\sigma_g^2(x-\mu_f)^2 - \sigma_f^2(x-\mu_g)^2}{2\sigma_f^2\sigma_g^2} \\[0.3cm]
 & = \frac{\sigma_g^2(x^2+\mu_f^2-2\mu_fx) - \sigma_f^2(x^2+\mu_g^2-2\mu_gx)}{2\sigma_f^2\sigma_g^2}
\end{split}
\end{equation}

Expandimos y ordenamos en base $x$,
\begin{equation}
\begin{split}
 \theta & = \left((\sigma_g^2 - \sigma_f^2)x^2 - 2(\sigma_g^2\mu_f - \sigma_f^2\mu_g)x + (\sigma_g^2\mu_f^2 - \sigma_f^2\mu_g^2 )\right) \frac{1}{2\sigma_f^2\sigma_g^2} \\[0.3cm]
 & = \left(x^2 - \frac{2(\sigma_g^2\mu_f - \sigma_f^2\mu_g)}{(\sigma_g^2 - \sigma_f^2)}x + \frac{(\sigma_g^2\mu_f^2 - \sigma_f^2\mu_g^2 )}{(\sigma_g^2 - \sigma_f^2)}\right) \frac{(\sigma_g^2 - \sigma_f^2)}{2\sigma_f^2\sigma_g^2}
\end{split}
\end{equation}

Esto es cuadr\'atico en x. Dado que un término $\varepsilon=0$, independiente de $x$ puede ser agregado para completar el cuadrado en $\theta$, esta prueba es suficiente para dterminar la media y la varianza cuando no es necesario normalizar.

\begin{equation}
 \sigma_{\div} = \sqrt{\frac{\sigma_f^2\sigma_g^2}{(\sigma_g^2 - \sigma_f^2)}}
\end{equation}

\begin{equation}
 \mu_{\div} = \frac{(\sigma_g^2\mu_f - \sigma_f^2\mu_g)}{(\sigma_g^2 - \sigma_f^2)}
\end{equation}

agregado $\varepsilon = \frac{\mu_{\div}^2-\mu_{\div}^2}{2\sigma_{\div}^2}$

\begin{equation}
\theta = \frac{x^2 - 2\mu_{\div} + \mu_{\div}^2 }{2\sigma_{\div}^2} + \underbrace{ \frac{ \frac{(\sigma_g^2\mu_f^2 - \sigma_f^2\mu_g^2)}{(\sigma_g^2 - \sigma_f^2)} - \mu_{\div}^2 }{2\sigma_{\div}^2} }_{\varphi}
\end{equation}

Reorganizando $\varphi$
\begin{equation}
\begin{split}
 \varphi & = \left( \frac{(\sigma_g^2\mu_f^2 - \sigma_f^2\mu_g^2)}{(\sigma_g^2 - \sigma_f^2)} - \left(\frac{(\sigma_g^2\mu_f - \sigma_f^2\mu_g)}{(\sigma_g^2 - \sigma_f^2)} \right)^2 \right) \frac{(\sigma_g^2 - \sigma_f^2)}{2\sigma_f^2\sigma_g^2} \\[0.3cm]
 & = \left((\sigma_g^2\mu_f^2 - \sigma_f^2\mu_g^2)(\sigma_g^2 - \sigma_f^2) - \left((\sigma_g^2\mu_f - \sigma_f^2\mu_g) \right)^2 \right) \frac{1}{2\sigma_f^2\sigma_g^2(\sigma_g^2 - \sigma_f^2)} \\[0.3cm]
 & =  \left( \cancel{\sigma_g^4\mu_f^2} - 2\sigma_f^2\sigma_g^2\mu_g^2 + \bcancel{\sigma_f^4\mu_g^2} - (\cancel{\sigma_g^4\mu_f^2} + \bcancel{\sigma_f^4\mu_g^2 } - 2\sigma_f^2\sigma_g^2\mu_f\mu_g)\right) \frac{1}{2\sigma_f^2\sigma_g^2(\sigma_g^2 - \sigma_f^2)}
 \end{split}
\end{equation}

Cancelando los $\sigma^2$
\begin{equation}
 \varphi = \frac{- \mu_g^2 - \mu_f^2 + 2\mu_f\mu_g}{2(\sigma_g^2 - \sigma_f^2)} = \frac{- (\mu_g - \mu_f)^2}{2(\sigma_g^2 - \sigma_f^2)}
\end{equation}

Luego $\theta$
\begin{equation}
 \theta = \frac{(x - \mu_{\div})^2}{2\sigma_{\div}^2} - \frac{(\mu_g - \mu_f)^2)}{2(\sigma_g^2 - \sigma_f^2)} 
\end{equation}

Por lo tanto
\begin{equation}
\begin{split}
 \kappa & = \frac{\sigma_g}{\sigma_f}  \, \text{exp}\left(- \frac{(x - \mu_{\div})^2}{2\sigma_{\div}^2} + \frac{(\mu_g - \mu_f)^2)}{2(\sigma_g^2 - \sigma_f^2)}  \right)\\[0.3cm]
 & = \frac{\sigma_g}{\sigma_f} \, e^{-\frac{(x - \mu_{\div})^2}{2\sigma_{\div}^2}} \, e^{\frac{(\mu_g - \mu_f)^2)}{2(\sigma_g^2 - \sigma_f^2)}}
\end{split}
\end{equation}

Multiplicando por $\frac{\sqrt{2\pi}}{\sqrt{2\pi}}\frac{\sigma_{\div}}{\sigma_{\div}}\frac{\sqrt{\sigma_g^2 - \sigma_f^2}}{\sqrt{\sigma_g^2 - \sigma_f^2}}=1$,
\begin{equation}
\begin{split}
 \kappa & =  \frac{1}{\sqrt{2\pi}\sigma_{\div}} \, e^{-\frac{(x - \mu_{\div})^2}{2\sigma_{\div}^2}} \, \left( \frac
 {1}{\sqrt{2\pi(\sigma_g^2 - \sigma_f^2)} } e^{-\frac{(\mu_g - \mu_f)^2)}{2(\sigma_g^2 - \sigma_f^2)}} \right)^{-1} \, \frac{\sigma_{\div}}{\sqrt{\sigma_g^2 - \sigma_f^2}}\frac{\sigma_g}{\sigma_f}\\[0.3cm]
 & = \frac{N\left(x; \mu_{\div},\sigma_{\div}\right)}{N\left(\mu_g;\mu_f,\sigma_g^2-\sigma_f^2\right)} \frac{\sigma_g^2}{\sigma_g^2 - \sigma_f^2}
\end{split}
\end{equation}




\end{document}
